% Standalone document
\documentclass[notes.tex]{subfiles}
\begin{document}
%%%%%%%%%%%%%%%%%%%%%%%%%%%%%%%%%%%%%%%%%%%%%%%%%%%%%%%
%%%%%%%%%%%%%%%%%%%%%%%%%%%%%%%%%%%%%%%%%%%%%%%%%%%%%%%
\chapter{Superspace}
\label{chap:superspace}
%%%%%%%%%%%%%%%%%%%%%%%%%%%%%%%%%%%%%%%%%%%%%%%%%%%%%%%
%%%%%%%%%%%%%%%%%%%%%%%%%%%%%%%%%%%%%%%%%%%%%%%%%%%%%%%
In this chapter we will introduce a very handy notation system for considering supersymmetry transformations effected by the $Q$ elements of the superalgebra, or, more correctly, the elements of the super-Poincaré group and their representations. This notation uses a coordinate system called superspace, and allows us to define so-called superfields as a replacement of ordinary field theory fields. This mirrors the Lorentz invariance built into relativistic field theory by using four-vectors. In order to do this we need to know a little more about the properties of Grassman (anti-commuting) numbers. However, we begin by taking another look at the familiar four-vectors in light of what we have learnt about continuous groups and Lie algebras.


%%%%%%%%%%%%%%%
\section{An initial skirmish: four-vectors as a coset space}
\label{sec:four-vectors}
%%%%%%%%%%%%%%%
Traditionally, we are introduced to four-vectors as a record keeping device for time and space-position in Special Relativity. In this notation we introduce (greek) four-vector indices, and some (odd) rules for manipulating these. 

Let us now go back to the Poincaré group and its generators in Sec.~\ref{sec:Poincare_group}, and in particular the exponential map in Eq.~(\ref{eq:Poincare_exp_map}). We here followed the conventions of four-vectors, but we could have equally well written it up using ordinary vector component notation, starting with the generators we derived for the different transformations of the group, {\it i.e.} rotations $J_i$, boosts $K_i$, time-translation $P_0$, and space-translation $P_i$. 


We can now form the (right) coset of the Poincaré group $P$ with its Lorentz subgroup $L$, $P/L$. This is not a group since $L$ is not normal to $P$, however, the coset space is a vector space formed by the elements $\{ \Lambda g | \Lambda \in L\}$ where $g\in P$. As before, we parameterise a general element $g$ in the Poincaré group $P$ as
\begin{equation*}
g=\exp\left(\frac{i}{2}\omega^{\rho \sigma}M_{\rho \sigma}+ia^\mu P_\mu\right),
\end{equation*}
but keeping in mind the discussion above. Since any two cosets are either disjoint or identical, we can now represent each element in the coset space by one of the members in the coset picking the member where $\omega^{\rho \sigma}=0$ giving $\Lambda=I$. What remains in the coset is the four-dimensional translation operator $T(a)=e^{ia^\mu P_\mu}$.  Physically, this space is the set of all translations independent of reference frames (boosts and rotations), and it is equivalent (isomorphic) to the vector space formed by the four components of $a^\mu$ (parameters of the translation) since there is a one-to-one map between $a^\mu$  and the elements of the coset space. Thus we can say that the four-vectors form the coset space between the Poincaré group and the Lorentz group.

The transformation properties of four-vectors that we have learnt about can be demonstrated by the properties of the Poincaré group. It is possible to show, using only the properties of the generators and their commutators,  that with a member of the Lorentz group $\Lambda=\exp(\frac{i}{2}\omega^{\rho \sigma}M_{\rho \sigma})$ we have
\[ \Lambda T(a) = \exp\left(\frac{i}{2}\omega^{\rho \sigma}M_{\rho \sigma}\right)T(a)=T(\Lambda a) \exp\left(\frac{i}{2}\omega^{\rho \sigma}M_{\rho \sigma}\right) =T(\Lambda a) \Lambda, \]
where $\Lambda a \equiv \Lambda^\mu_{~\nu} a^\nu$ is the same Lorentz transformation of the four-vector translation parameter as the one we are accustomed to.

If we now want to see how a Lorentz transformation $\Lambda$ acts for example on a scalar function $F(x)$  we can write $F(x)=\exp(ix^\mu P_\mu)F(0)$. This means that the transformation is essentially captured as $x'$ in $\Lambda\exp(ix^\mu P_\mu)= \exp(ix'^\mu P_\mu)$. Now, from the above we have
\[ \Lambda\exp(ix^\mu P_\mu)=\Lambda T(x)=T(\Lambda x) \Lambda, \]
and since the Lorentz transformation does nothing to a scalar coordinate independent quantity such as $F(0)$, we have $x'^\mu=\Lambda^\mu_{~\nu} x^\nu$, and the scalar function transforms the way we are used to under Lorentz transformations, namely $\Lambda F(x)= F(x')$. This argument can easily be extended to vector fields $A_\mu(x)$ and higher rank tensors, for example
\[ \Lambda A^\nu(x)= \Lambda^\mu_{~\nu} A^\nu(x'). \]
 The way we have defined four-vectors have built the Lorentz transformations of Special Relativity into our equations.
 
 %TODO: Mention Erlangen programme

%\begin{eqnarray*}
%\Lambda \exp(ix^\mu P_\mu) &=& \exp\left(\frac{i}{2}\omega^{\rho \sigma}M_{\rho \sigma}\right)\exp(ix^\mu P_\mu) \\
%&=& \exp\left(\frac{i}{2}\omega^{\rho \sigma}M_{\rho \sigma}+ix^\mu P_\mu  - \frac{1}{4} \omega^{\rho \sigma}x^\mu [M_{\rho \sigma},P_\mu ]+\ldots\right)
%\end{eqnarray*}
%using the Baker-Campbell-Haussdorff expression from (\ref{eq:BKH_matrices}) and the fact that $P_\mu$ commutes with itself from (\ref{eq:poco2}). Here, all higher order terms contain the commutator between $M_{\mu\nu}$ and $P_\mu$
%\begin{equation}
%\frac{1}{4} \omega^{\rho \sigma}x^\mu [M_{\rho \sigma},P_\mu ] = \frac{i}{4} \omega^{\rho \sigma}x^\mu (g_{\rho\mu} P_\sigma - g_{\sigma \mu} P_\rho)
%= \frac{i}{4}( \omega^{\rho \sigma}x_\rho P_\sigma - \omega^{\rho \sigma}x_\sigma  P_\rho)
%= \frac{i}{2}\omega^{\rho \sigma}x_\rho P_\sigma
%\end{equation}



%%%%%%%%%%%%%%%%%%%%%%%%%%%%%%%%%%%%%%%%%%%%%%%%
\section[Superspace definition]{Superspace definition}
\label{sec:superspace}
%%%%%%%%%%%%%%%%%%%%%%%%%%%%%%%%%%%%%%%%%%%%%%%%
Superspace\footnote{First introduced by Salam \& Strathdee~\cite{Salam:1974jj}.} is a coordinate system where supersymmetry transformations are {\it manifest}, in other words, the action of elements in the super-Poincaré group ($SP$) are treated like Lorentz-transformations are in Minkowski space.
\df{{\bf Superspace} is an eight-dimension manifold that can be constructed from the {\bf coset space} of the super-Poincaré group, $SP$, and the Lorentz group, $L$, $SP/L$, by giving coordinates $z^\pi = (x^\mu, \theta^A, \bar{\theta}_{\dot{A}})$, $\pi=\mu,A,\dot A$, where $x^\mu$ are the ordinary Minkowski coordinates, and where $\theta^A$ and $\bar{\theta}_{\dot{A}}$ are four Grassman (anti-commuting) numbers in the form of Weyl spinors, being the parameters of the supercharges $Q_A$ and $\bar{Q}^{\dot{A}}$, respectively, in the exponential map of the superalgebra.}

To understand what all of this means we start from the same perspective as for the four-vectors in the previous section, and begin by writing a general element of SP, $g\in SP$, using the exponential map defined in Section \ref{sec:lie_algebras}:
\[g = \exp\left(ix^\mu P_\mu + i\theta^{A} Q_A + i\bar{\theta}_{\dot{A}}\bar{Q}^{\dot{A}} + \frac{i}{2}\omega_{\rho\nu}M^{\rho\nu}\right),\]
where $x^\mu$, $\theta^{A}$, $\bar{\theta}_{\dot{A}}$ and $\omega_{\rho\nu}$ constitute the parametrisation of the group, and $P_\mu$, $Q_A$, $\bar Q^{\dot{A}}$ and $M_{\rho\nu}$ are the generators. Following the same argument as above we can now parametrise the coset space $SP/L$ simply by setting $\omega_{\mu\nu} = 0$.\footnote{$SP/L$ is again {\it not} a coset group, because $L$ is not a normal subgroup of $SP$, but it still forms a vector space (the coset space) which we call superspace.} The remaining parameters of $SP/L$ are then the coordinates that span superspace. 

In order for the exponential map to make sense the parameters $\theta$ need to anti-commute just like the $Q$s so that the contractions $\theta^{A} Q_A$ and $\bar{\theta}_{\dot{A}}\bar{Q}^{\dot{A}}$ follow ordinary commutation rules. Otherwise the commutation properties of the exponential map would be different order by order.\footnote{We might already see how this can be useful: if we consistently use $\theta^AQ_A$ and $\bar{\theta}_{\dot{A}}\bar{Q}^{\dot{A}}$ instead of only $Q_A$ and $\bar{Q}^{\dot{A}}$ in Eqs.~(\ref{eq:QPweyl})--(\ref{eq:QQbarweyl}) we can actually rewrite the superalgebra as an ordinary Lie algebra, but with Grassman elements, because of these commutation properties.}
 As physicists we also want to know the dimensions of our new parameters. To do this we first look at Eq.~(\ref{eq:QQbarweyl}):
\[\{Q_A, \bar{Q}_{\dot{B}}\} = 2\sigma^{\mu}_{A\dot{B}}P_\mu,\]
where we know that $P_\mu$ has mass dimension $[P_\mu] = M$. This means that $[Q^2] = M$ and $[Q]=M^{1/2}$. In the exponential, all terms must have mass dimension zero to make sense. This means that $[\theta Q] = 0$, and therefore $[\theta] = M^{-1/2}$.

We now want to find the effect of supersymmetry transformations (transformations by the super-Poincaré group) on the superspace coordinates, and we begin by noting that any $SP$ transformation can effectively be written in the following way without the boosts and rotations of the Lorentz group:
\[g_0(a,\alpha) = \exp[ia^\mu P_\mu + i\alpha^AQ_A +i\bar{\alpha}_{\dot{A}}\bar{Q}^{\dot{A}}],\]
effectively setting $\omega_{\rho\nu}=0$, because we can again show that\footnote{Fortunately we are not going to do this because it is messy, but it can be done using the algebra of the group and the series expansion of the exponential function. Note, however, that the proof rests on the $P$s and $Q$s forming a closed set, which we saw in the algebra Eqs.~(\ref{eq:QPweyl})--(\ref{eq:QQbarweyl}).}
\begin{equation}
\exp\left[\frac{i}{2}\omega_{\rho\nu}M^{\rho\nu}\right]g_0(a,\alpha) = g_0(\Lambda a,S(\Lambda)\alpha)\exp\left[\frac{i}{2}\omega_{\rho\nu}M^{\rho\nu}\right],
\end{equation}
{\it i.e.}\ that all that a Lorentz boost does is to transform spacetime coordinates by $\Lambda$ and Weyl spinors by $S(\Lambda)$, which is the spinor representation of $\Lambda$ ($SL(2,\mathbb C)$). Thus, in more colloquial terms, for the supersymmetry transformation it does not matter which reference frame we are working in, we know how the transformation changes between the frames, given by  $\Lambda a$  and $S(\Lambda)\alpha$. 

We can now find the transformation of superspace coordinates under a supersymmetry transformation, just as we have seen the Minkowski coordinates transform under Lorentz transformations. It might be tempting to look directly at the effects of an element $g_0(a, \alpha)$ on a function on superspace coordinates, $F(z^\pi) = F(x^\mu, \theta^A, \bar{\theta}_{\dot{A}})$, just as we did for the translation group. However, the powers of the sum of generators in the infinite series will become very unwieldy and messy. Instead, we pull out the coordinate dependence from the function $F(z^\pi) = e^{iz^\pi K_\pi}F(0)$, where $K_\pi=(P_\mu,Q_A,\bar{Q}^{\dot{A}})$,  and we look at the transformation $z^\pi \to z'{}^\pi$ given by
\[g_0e^{iz^\pi K_\pi} = e^{iz'{}^\pi K_\pi}.\]
In moving the problem to the exponential, we can use the power of some results on the exponentials of non-commuting quantities, in particular the Campbell-Baker-Hausdorff expansion from (\ref{eq:BKH_matrices}).
We have\footnote{Here we use Campbell-Baker-Hausdorff expansion $e^{A}e^{B} = e^{A + B + \frac{1}{2}[A, B] + ...}$ where the terms that follow all contain commutators of the first commutator  $[A, B] $ and the operators $A$ and $B$.}
\begin{eqnarray*}
g_0 e^{iz^\pi K_\pi} &=& \exp(ia^\nu P_\nu + i\alpha^BQ_B + i\bar{\alpha}_{\dot{B}}\bar{Q}^{\dot{B}})\exp( iz^\pi K_\pi)\\
&=& \exp(ia^\nu P_\nu + i\alpha^BQ_B + i\bar{\alpha}_{\dot{B}}\bar{Q}^{\dot{B}} + iz^\pi K_\pi\\
&& + \frac{1}{2}[ia^\nu P_\nu + i\alpha^BQ_B + i\bar{\alpha}_{\dot{B}}\bar{Q}^{\dot{B}}, iz^\pi K_\pi] + \ldots)
\end{eqnarray*}
Now we take a closer look at the commutator:\footnote{Using that $P_\mu$ commutes with all the other generators present, as well as $[\theta^A Q_A, \xi^B Q_B] = -\theta^A\xi^B\{Q_A, Q_B\} = 0$, and similarly for $\bar{Q}^{\dot{B}}$.}
\begin{eqnarray*}
[~,~]&=&-[\alpha^BQ_B,\bar{\theta}_{\dot{A}}\bar{Q}^{\dot{A}}] - [\bar{\alpha}_{\dot{B}}\bar{Q}^{\dot{B}}, \theta^AQ_A]\\
&=&\alpha^B\bar{\theta}_{\dot{A}}\epsilon^{\dot{A}\dot{C}}\{Q_B,\bar{Q}_{\dot{C}}\} + \bar{\alpha}_{\dot{B}}\theta^A\epsilon^{\dot{B}\dot{C}}\{\bar{Q}_{\dot{C}}, Q_A\}\\
&=&2\alpha^B\bar{\theta}_{\dot{A}}\epsilon^{\dot{A}\dot{C}}\sigma^\mu_{B\dot{C}}P_\mu + 2\bar{\alpha}_{\dot{B}}\theta^A\epsilon^{\dot{B}\dot{C}}\sigma^\mu_{A\dot{C}}P_\mu\\
&=&2(\alpha^B\bar{\theta}^{\dot{C}}\sigma^\mu_{B\dot{C}} + \bar{\alpha}^{\dot{C}}\theta^A\sigma^\mu_{A\dot{C}})P_\mu .
\end{eqnarray*}
We can relabel $B = A$ and $\dot{C} = \dot{A}$ which leads to 
\begin{eqnarray*}
\frac{1}{2}[~,~]&=&(\alpha^A\sigma^\mu{}_{A\dot{A}}\bar{\theta}^{\dot{A}} -\theta^A\sigma^\mu{}_{A\dot{A}}\bar{\alpha}^{\dot{A}})P_\mu.
\end{eqnarray*}
The commutator is proportional with $P_\mu$, and will therefore commute with all the operators in the problem, in particular the higher terms in the Campbell-Baker-Hausdorff expansion, meaning that the series reduces to
\begin{eqnarray*}
g_0 e^{iz^\pi K_\pi} 
&=& \exp[i(x^\mu + a^\mu - i\alpha^A\sigma^\mu_{A\dot{A}}\bar{\theta}^{\dot{A}} + i\theta^A\sigma^\mu_{A\dot{A}}\bar{\alpha}^{\dot{A}})P_\mu + i(\theta^A+\alpha^A)Q_A + i(\bar{\theta}_{\dot{A}}+\bar{\alpha}_{\dot{A}})\bar{Q}^{\dot{A}}].
%&=& \exp[i(x^\mu + a^\mu+ i\alpha\sigma^\mu\bar\theta - i\theta\sigma^\mu\bar\alpha)P_\mu + i(\theta^A+\alpha^A)Q_A + i(\bar{\theta}_{\dot{A}}+\bar{\alpha}_{\dot{A}})\bar{Q}^{\dot{A}}].
\end{eqnarray*}
So superspace coordinates transform under supersymmetry transformations as:
\begin{equation}
(x^\mu, \theta^A, \bar{\theta}_{\dot{A}}) \to  (x^\mu + a^\mu -i\alpha^A\sigma^\mu_{A\dot{A}}\bar{\theta}^{\dot{A}} + i\theta^A\sigma^\mu_{A\dot{A}}\bar{\alpha}^{\dot{A}}, \theta^A + \alpha^A, \bar{\theta}_{\dot{A}} + \bar{\alpha}_{\dot{A}}),
\end{equation}
or given more explicitly as a composition function
\begin{equation}
f_\pi(x^\mu, \theta^A, \bar{\theta}_{\dot{A}},a^\mu, \alpha^A, \bar{\alpha}_{\dot{A}}) =  (x^\mu + a^\mu -i\alpha^A\sigma^\mu{}_{A\dot{A}}\bar{\theta}^{\dot{A}} + i\theta^A\sigma^\mu{}_{A\dot{A}}\bar{\alpha}^{\dot{A}}, \theta^A + \alpha^A, \bar{\theta}_{\dot{A}} + \bar{\alpha}_{\dot{A}}).
\end{equation}

As a crucial by-product we can now write down a differential representation for the supersymmetry generators by applying the standard expression for the generators $X_i$ of a Lie algebra, given the composition functions $f_\pi$:
\[ iX_j = \frac{\partial f_\pi}{\partial a_j}\frac{\partial}{\partial z_\pi},\]
which gives us:
\begin{eqnarray}
iP_\mu &=& \partial_\mu \label{eq:diffrepP},\\
iQ_A &=& -i(\sigma^\mu\bar{\theta})_A\partial_\mu +\partial_A,\\
i\bar{Q}^{\dot{A}} &=& -i(\bar{\sigma}^\mu \theta)^{\dot{A}}\partial_\mu +\partial^{\dot{A}}.
\label{eq:diffrepQbar}
\end{eqnarray}
The interested reader can now use these expressions to check the (anti-)commutation relations for the supercharges in Eqs.~(\ref{eq:QQweyl}) and (\ref{eq:QQbarweyl}).


%%%%%%%%%%%%%%%
\section{Superspace calculus}
\label{sec:calc}
%%%%%%%%%%%%%%%
It should be clear from the following section that we need to know something about the calculus of anti-commuting Grassmann numbers in order to make sense of differentiation (and integration) with respect to them. In this section we will briefly discuss their most important properties, focusing on the coordinates of superspace.

As Grassmann numbers the superspace coordinates obey the following commutation rules:
\[\{\theta_A, \theta_B\} = \{\theta_A, \bar{\theta}_{\dot{B}}\} = \{ \bar{\theta}_{\dot{A}}, \theta_B\} = \{\bar{\theta}_{\dot{A}}, \bar{\theta}_{\dot{B}}\} =0.\]
From this we get the relationships:\footnote{There is no summation implied in the first two lines, only a repetition of the same superspace coordinate. These are of course the same relations we already used for the Weyl spinors.}
\begin{eqnarray}
\theta_A^2 &=& \theta_A \theta_A = -\theta_A \theta_A = 0, \label{eq:thetasq}\\
\bar{\theta}_{\dot{A}}^2 &=& \bar{\theta}_{\dot{A}}\bar{\theta}_{\dot{A}} = -\bar{\theta}_{\dot{A}}\bar{\theta}_{\dot{A}} = 0, \\
\theta^2 &\equiv& \theta\theta \equiv \theta^A\theta_A = -2\theta_1\theta_2,\\
\bar{\theta}^2 &\equiv& \bar{\theta}\bar{\theta} \equiv \bar{\theta}_{\dot{A}}\bar{\theta}^{\dot{A}} = 2\bar{\theta}^{\dot{1}}\bar{\theta}^{\dot{2}}.
\end{eqnarray}
Notice that if we have a function $f$ of a Grassman number, say $\theta_A$, then the all-order expansion of that function in terms of $\theta_A$, is
\begin{equation}
f(\theta_A) = a_0 + a_1 \theta_A,\label{eq:fexp}
\end{equation}
as there are simply no more terms because of (\ref{eq:thetasq}).

We now need to define differentiation and integration on these numbers in order to create a calculus for them.\footnote{These definitions have no infinitesimal interpretations, nor is there any separate notion of definite integrals.}
\df{We define differentiation in superspace  by:
\[\partial_A \theta^B \equiv \frac{\partial}{\partial \theta^A} \theta^B \equiv \delta_A{}^B,\]
with a product rule
\begin{eqnarray}
\partial_A(\theta^{B_1} \theta^{B_2}\theta^{B_3}\ldots\theta^{B_n} ) &\equiv& (\partial_A\theta^{B_1}) \theta^{B_2}\theta^{B_3}\ldots\theta^{B_n}\nonumber\\
&&- \theta^{B_1}(\partial_A\theta^{B_2})\theta^{B_3}\ldots\theta^{B_n}\nonumber\\
 && +\ldots+(-1)^{n-1}\theta^{B_1}\theta^{B_2}\ldots(\partial_A\theta^{B_n}).
 \end{eqnarray}}
 This implies that the differential operator $\partial_A$ is itself a Grassmann number and anti-commutes. We can show that the indices of these operators can be raised and lowered with the $\epsilon$ in Eqs.~(\ref{eq:epsilonAB}) and (\ref{eq:epsilonAdotBdot}).

\df{The integral of a function $f$ of a superspace coordinate $\theta_A$ is defined as a functional $I[f]$
\[ I[f]=\int d\theta_A \, f(\theta_A), \]
which evaluates to a {\bf $\mathbf c$-number} (number in $\mathbb C$). The evaluation is defined by the relations
 \[\int d\theta_A \equiv 0,\quad \int d\theta_A \, \theta_A \equiv 1,\] 
and similarly for $\bar\theta_{\dot A}$, and we demand linearity:
\[\int d\theta_A[af(\theta_A) + bg(\theta_A)] \equiv a\int d\theta_A f(\theta_A) + b\int d\theta_A g(\theta_A).\]
In these definitions there is no implied summation over the index $A$.}

This integral definition has a surprising property. If we take the integral of (\ref{eq:fexp}) we get:
\[\int d\theta_A f(\theta_A) = a_1 = \partial^A f(\theta_A),\]
meaning that differentiation and integration has the same effect on functions of Grassmann numbers.

To integrate over multiple Grassmann numbers we define volume elements as
\begin{eqnarray*}
d^2\theta &\equiv& -\frac{1}{4}d\theta^Ad\theta_A,\\
d^2\bar{\theta} &\equiv& -\frac{1}{4}d\bar{\theta}_{\dot{A}}d\bar{\theta}^{\dot{A}},\\
d^4\theta &\equiv& d^2\theta d^2\bar{\theta},
\end{eqnarray*}
and we demand that the integral operators anti-commute, just as the differential operators
\[ \{\int d\theta_A,\int d\theta_B\}=\{\int d\theta_A,\theta_B\}=0.\]
This specific volume element definition is made to normalise the following integrals
\begin{eqnarray*}
\int d^2\theta\text{ }\theta\theta &=& 1,\\
\int d^2\bar{\theta}\text{ }\bar{\theta} \bar{\theta} &=& 1,\\
\int d^4 \theta \text{ }(\theta\theta)(\bar{\theta} \bar{\theta}) &=& 1.
\end{eqnarray*}

Delta functions of Grassmann variables are given by:
\[\delta(\theta_A) = \theta_A,\]
\[\delta^2(\theta_A) = \theta\theta,\]
\[\delta^2(\bar{\theta}^{\dot{A}}) = \bar{\theta}\bar{\theta},\]
and we can easily show that these functions satisfy, just as the usual definition of delta functions,
\[\int d\theta_A f(\theta_A)\delta(\theta_A) = f(0).\]



%%%%%%%%%%%%%%%
\section{Covariant derivatives}
\label{sec:covder}
%%%%%%%%%%%%%%%
Similar to the properties of covariant derivatives for gauge transformations in gauge theories, it would be nice to have a derivative that is invariant under supersymmetry transformations, {\it i.e.}\ commutes with the supersymmetry generators. Obviously $P_\mu = -i\partial_\mu$ does this, but more general covariant derivatives can be made.
\df{The following {\bf covariant derivatives} commute with supersymmetry transformations:
\begin{eqnarray}
D_A &\equiv& \partial_A + i(\sigma^\mu\bar{\theta})_A\partial_\mu ,\\
\bar{D}_{\dot{A}} &\equiv& -\partial_{\dot{A}}-i(\theta\sigma^\mu)_{\dot{A}}\partial_\mu.
\end{eqnarray}}

These can be shown to satisfy the following relations that are useful in calculations:
\begin{eqnarray}
\{D_A, D_B\} &=& \{\bar{D}_{\dot{A}}, \bar{D}_{\dot{B}}\} = 0,\\
\{D_A, \bar{D}_{\dot{B}}\} &=& -2\sigma^\mu_{A\dot{B}}P_\mu, \label{eq:D2}\\
D^3 = \bar{D}^3&=&0 \label{eq:D3}, \\
D^A\bar{D}^2 D_A &=& \bar{D}_{\dot{A}}D^2\bar{D}^{\dot{A}}.
\end{eqnarray}
Here, $D^3$ and $\bar D^3$ means the application of at least three of these covariant derivatives.

From the covariant derivatives we can also construct  a set of three projection operators.
\df{The operators
\begin{eqnarray}
\pi_+ &\equiv& -\frac{1}{16\Box} \bar{D}^2D^2,\\
\pi_- &\equiv& -\frac{1}{16\Box} D^2\bar{D}^2,\\
\pi_T &\equiv& \frac{1}{8\Box} \bar{D}_{\dot{A}}D^2\bar{D}^{\dot{A}},
\end{eqnarray}
with $\Box\equiv\partial_\mu\partial^\mu$, are orthogonal projection operators, {\it i.e.}\ they fulfil:
\begin{eqnarray}
\pi_{\pm,T}^2 &=&\pi_{\pm, T}, \\
\pi_+\pi_- &=& \pi_+\pi_T = \pi_-\pi_T= 0,\\
\pi_+ + \pi_- + \pi_T &=& 1.
\end{eqnarray}
}



%%%%%%%%%%
\section{Superfields}
%%%%%%%%%%
Using the superspace coordinates we can now define functions of these to use in a field theory. Naturally we should call these objects superfields.
\df{A {\bf superfield} $\Phi$ is an operator valued function on superspace $\Phi(x, \theta, \bar{\theta})$.}
Notice how we write, just as for ordinary fields depending on four-vector coordinates, the coordinates sans indices. 

We can expand any such superfield $\Phi(x, \theta, \bar{\theta})$  as a power series in $\theta$ and $\bar{\theta}$. For a superfield without explicit spinor indices this gives in general,
%\footnote{Equation (\ref{eq:gen_sup}) can be shown to be closed under supersymmetry transformations, meaning that a superfield transforms into another superfield under the transformations of the previous section.}
\begin{eqnarray}
\Phi(x, \theta, \bar{\theta}) &=& f(x) + \theta^A\phi_A(x) + \bar{\theta}_{\dot{A}}\bar{\chi}^{\dot{A}}(x) + \theta\theta m(x) + \bar{\theta}\bar{\theta}n(x) \nonumber\\ 
&& + \theta \sigma^\mu \bar{\theta}V_\mu(x) + \theta \theta \bar{\theta}_{\dot{A}}\bar{\lambda}^{\dot{A}}(x) +\bar{\theta}\bar{\theta}\theta^A\psi_A(x) + \theta\theta\bar{\theta}\bar{\theta}d(x),\label{eq:gen_sup}
\end{eqnarray} 
where the functions of space-time coordinates $x$ are called the {\bf component fields}. Any superfield without explicit spinor indices, such as the one above, commutes with any other superfield, because all the Grassmann numbers appear in contracted pairs. 

The properties of the component fields can be deduced from the requirement that $\Phi$ must be an (operator valued) Lorentz scalar or pseudoscalar. These are shown in Table~\ref{tab:gensup} along with the corresponding degrees of freedom each field has.
\begin{center}
   \begin{tabular}{c |l| c} 
   \noalign{\smallskip}\hline\noalign{\smallskip}
   {\bf Component field} & {\bf Type} & {\bf d.o.f.} \\
   \noalign{\smallskip}\hline\noalign{\smallskip}
   $f(x)$, $m(x)$, $n(x)$ & Complex (pseudo) scalar & 2\\
   $\psi_A(x)$, $\phi_A(x)$ & Left-handed Weyl spinor & 4\\
   $\bar{\chi}^{\dot{A}}(x)$, $\bar{\lambda}^{\dot{A}}(x)$ &Right-handed Weyl spinor& 4\\
   $V_\mu(x)$ & Lorentz 4-vector (complex) & 8\\
   $d(x)$ & Complex scalar & 2 \\
   \noalign{\smallskip}\hline\noalign{\smallskip}
    \end{tabular}
   \captionof{table}{Component field content of a general superfield. \label{tab:gensup}}
   \end{center}

One can show that under supersymmetry transformations these component fields transform linearly into each other, thus we say that superfields (with the differential form of the supersymmetry generators) are {\it representations} of the super-Poincar\'e group -- in the sense of being states in a representation space -- just as ordinary quantum fields are representations of the Poincaré group, albeit {\it highly reducible} representations since there are subsets of the component fields that are closed on the supersymmetry transformation!\footnote{Indeed, they are linear representations since a sum of superfields is a superfield, and the differential supersymmetry operators act linearly.} 
%We often write the supersymmetry transformation as
%\[ \delta_S=(i\theta Q+i\bar\theta \bar Q), \] 
%as the translation part is `trivial' and its generators in any case commutes with the $Q$s. 
For example, the scalar fields $f$ and $m$, and the Weyl-spinor $\phi$ transform as\footnote{A word of warning is in order here. Considering $\Phi$ as a {\it quantum} field with operator value it transforms under the supersymmetry transformations in the {\it passive sense}  
\[ \Phi(x)\to \exp{(-i\alpha Q-i\bar\alpha \bar Q)}\Phi(x)\exp{(i\alpha Q+i\bar\alpha \bar Q)}. \]}
\begin{eqnarray}
\delta_S f &=& \alpha\phi +\bar\alpha\bar\chi, \\
\delta_S\phi_A &=& 2 \alpha_A m + (\sigma^\mu \bar\alpha)_A (i\partial_\mu f + V_\mu), \\
\delta_S m &=& \bar\alpha\bar\lambda-\frac{i}{2}\partial_\mu\phi\sigma^\mu\bar\alpha .
\label{eq:general_superfield_transform}
\end{eqnarray}
% TODO: Add some more examples of the actual transforms here

We can recover the known irreducible representations, see Section~\ref{sec:superalgebrarep}, by imposing some restrictions on the fields. To do this we define the following three types of superfields that we will discuss below:
\begin{eqnarray}
\bar{D}_{\dot{A}}\Phi(x, \theta, \bar{\theta}) =& 0 &\quad\text{({\bf left-handed scalar superfield})}\label{eq:leftsup}\\
D_{A}\Phi^\dagger (x, \theta, \bar{\theta}) =& 0 &\quad\text{(\bf{right-handed scalar superfield})}\label{eq:rightsup}\\
\Phi^\dagger (x, \theta, \bar{\theta}) =& \Phi(x, \theta, \bar{\theta}) &\quad\text{({\bf vector superfield})}
\end{eqnarray}
Note that it is $\Phi^\dagger$ which is the right-handed superfield in Eq.~(\ref{eq:rightsup}), not $\Phi$. However, we can show that the hermitian conjugate of a left-handed scalar superfield fulfils the condition for a right-handed scalar superfield. We always keep the ``dagger'' operator on the right-handed fields to remember what they are since the difference between left- and right-handed superfields will become crucial later. Supersymmetry transformations can also be shown to transform left-handed superfields into left-handed superfields, right-handed superfields into right-handed superfields and vector superfields into vector superfields. This means that they are separate representations.

Products of the same type of superfield is a superfield of the same type since for left-handed scalar superfields $\Phi_i$ and $\Phi_j$,
\[ \bar{D}_{\dot{A}}(\Phi_i\Phi_j) = (\bar{D}_{\dot{A}}\Phi_i)\Phi_j +\Phi_i( \bar{D}_{\dot{A}}\Phi_j)=0,\]
and similarly for a right-handed scalar superfields, and for vector superfields $\Phi_i$ and $\Phi_j$,
\[(\Phi_i\Phi_j)^\dagger=\Phi_j^\dagger \Phi_i^\dagger=\Phi_j \Phi_i= \Phi_i\Phi_j.\]
The product of a left-handed scalar superfield $\Phi$ and its hermitian conjugate $\Phi^\dagger$, $V=\Phi \Phi^\dagger$, is a vector superfield since
\[V^\dagger=(\Phi \Phi^\dagger)^\dagger=\Phi \Phi^\dagger=V.\]
The same is true for sums of superfields of the same type. These properties will be important when creating a superfield version of a Lagrangian.
% TODO: Extend this to the idea of holonomic functions

Note that the projection operators that we defined in Section~\ref{sec:covder}, $\pi_\pm$, project out left-/right-handed superfields, respectively,  from the general superfield, because:
\[\bar{D}_{\dot{A}}\pi_+\Phi = D_A \pi_- \Phi^\dagger = 0, \]
which follows from using Eq.~(\ref{eq:D3}). This is analogous to the properties of $P_{L/R} = \frac{1}{2}(1\mp \gamma_5)$ that we saw in Sec.~\ref{sec:Dirac_spinors}.

%%%
\subsection{Scalar superfields}
%%%
What is the connection between the scalar superfields and the $j=0$ irreducible representation? To see this we use a cute trick:\footnote{Here cute is used in the widest possible sense.} Change to the variable $y^\mu \equiv x^\mu + i\theta\sigma^\mu \bar{\theta}$. Then the covariant derivatives simplify to
\begin{eqnarray}
D_A &=& \partial_A + 2i(\sigma^\mu\bar{\theta})_{A}\frac{\partial}{\partial y^\mu},\\
\bar{D}_{\dot{A}} &=& -\partial_{\dot{A}}.
\end{eqnarray}
This means that a field fulfilling $\bar{D}_{\dot{A}} \Phi = 0$ in the new set of coordinates must be independent of the $\bar{\theta}$ coordinates. Thus we can write this field as:
\begin{equation}
\Phi(y, \theta) = A(y) + \sqrt{2}\theta\psi(y) + \theta\theta F(y),
\label{eq:leftscalarsuperfield_y}
\end{equation}
and looking at the much restricted component field content we get the result in Table~\ref{tab:lscsup}. 
\begin{center}
   \begin{tabular}{c |l| c} 
   \noalign{\smallskip}\hline\noalign{\smallskip}
   {\bf Component field} & {\bf Type} & {\bf d.o.f.} \\
   \noalign{\smallskip}\hline\noalign{\smallskip}
   $A(x)$, $F(x)$ & Complex scalar & 2\\
   $\psi_A(x)$ & Left-handed Weyl spinor & 4\\
   \noalign{\smallskip}\hline\noalign{\smallskip}
    \end{tabular}
   \captionof{table}{Component fields contained in a left-handed scalar superfield. \label{tab:lscsup}}
   \end{center}

Since we wish to interpret the Weyl spinor here as a fermion quantum field with dimension $M^{3/2}$, and given that $[\theta]=M^{-1/2}$, the scalar superfield itself must have mass dimension $[\Phi]=1$. This means that the scalar field $A$ has the expected mass dimension $M^1$ of an ordinary scalar quantum field, however, the scalar $F$ has the odd mass dimension $[F]=2$.\footnote{Odd as in strange, 2 is known to be an even number.} The Weyl spinor is used to represent (half of) one of the Dirac fermions of the Standard Model, while the scalar $A$ is typically given the same name as the fermion with an `s'-prefix, and the scalar $F$ is called an {\bf auxiliary} field, for reasons which will become clear later.

We can undo the coordinate change in (\ref{eq:leftscalarsuperfield_y}) by inserting for $y$ and expanding in powers of $\theta$ and $\bar{\theta}$, giving
\begin{equation}
\Phi(x, \theta, \bar{\theta}) = A(x) + i(\theta\sigma^\mu \bar{\theta})\partial_\mu A(x) - \frac{1}{4}\theta\theta\bar{\theta}\bar{\theta}\Box A(x) + \sqrt{2}\theta \psi(x) - \frac{i}{\sqrt{2}}\theta\theta\partial_\mu \psi(x)\sigma^\mu\bar{\theta} + \theta\theta F(x).
\label{eq:leftscalarsuperfield}
\end{equation}

By using the transformation  $y^\mu \equiv x^\mu - i\theta\sigma^\mu \bar{\theta}$ we can show a similar field content for the right-handed scalar superfield. The general form of a right-handed scalar superfield is then as could be expected:
\begin{equation}
\Phi^\dagger (x, \theta, \bar{\theta}) = A^*(x) - i(\theta\sigma^\mu \bar{\theta})\partial_\mu A^*(x) - \frac{1}{4}\theta\theta\bar{\theta}\bar{\theta}\Box A^*(x) + \sqrt{2}\bar{\theta}\bar{\psi}(x) + \frac{i}{\sqrt{2}}\bar{\theta}\bar{\theta}\theta \sigma^\mu\partial_\mu \bar{\psi}(x) +\bar{\theta}\bar{\theta} F^*(x).
\label{eq:rightscalarsuperfield}
\end{equation}

We can now compare the above to the $j=0$ representation of the super-Poincaré group that had two scalar states and two fermionic states (d.o.f.).  After applying the equations of motions (e.o.m.)\ the auxiliary field $F(x)$, with the strange mass dimension, can be completely eliminated as it does not have any derivatives.\footnote{Remember that the classical equation of motion for a field/particle, the Lagrange equation, has a term with the derivative of the Lagrangian w.r.t.\ to the field derivatives/velocities and one w.r.t.\ the fields/coordinates. For $F$ only the latter is non-zero and can be used to solve for $F$. We will show this in more detail in Sec.~\ref{sec:eom} when we construct supersymmetric Lagrangians.} The e.o.m.\ also eliminate two of the fermion d.o.f. using the Direc/Weyl equations from Sec.~\ref{sec:Dirac_spinors}.\footnote{Since the Lagrangian is linear in time derivatives of $\psi$ -- it will have to be  when we construct it from these superfields that are linear -- the generalised momenta $\pi=\partial_0\psi$ can be re-expressed in terms of the generalised coordinates without time derivatives and are not independent coordinates.} 
% TODO: Move this comment to the Weyl spinor section and include a Lagrangian?
This does not happen for the scalar $A(x)$ since their e.o.m\ are not linear in the time-derivative. Thus, after the equations of motion, we are left with the same states as in the $j=0$ representation.

However, the scalar superfields will not correspond directly to particle states for the known Standard Model particles since, as we discussed in Sec.~\ref{sec:SP_irreps}, a Weyl spinor on its own cannot describe a Dirac fermion. When we construct particle representations we will take one left-handed scalar superfield and one {\it different} right-handed scalar superfield. These will form a Dirac fermion and two scalars (and their anti-particles) after application of the e.o.m. 

%%%
\subsection{Vector superfields}
%%%
If we take the general superfield in (\ref{eq:gen_sup}) and compare the $\Phi$ and $\Phi^\dagger$ expressions we can see that the following is the restricted component field structure of a vector superfield:
\begin{eqnarray*}
V(x, \theta, \bar{\theta}) &=& C(x) + \theta\phi(x) + \bar{\theta}\bar{\phi}(x) + \theta\theta M(x) + \bar{\theta}\bar{\theta}M^*(x) \\
&& + \theta \sigma^\mu \bar{\theta}V_\mu(x) + \theta \theta \bar{\theta}\bar{\lambda}(x) +\bar{\theta}\bar{\theta}\theta\lambda(x) + \theta\theta\bar{\theta}\bar{\theta}D(x).
\end{eqnarray*} 
The properties of the component fields  are summarised in Table~\ref{tab:vecsup}. Repeating the arguments of mass dimension, if the $V_\mu$ is to represent a vector quantum field with $[V_\mu]=M^1$, then the dimension of the vector superfield must be $[V]=M^0$, the Weyl-spinor $\lambda$ is a normal looking $[\lambda]=M^{3/2}$, while the scalar $D$ again has the odd $[D]=M^2$. For the $C$ and $\phi$ fields the mass dimension is even stranger.


\begin{center}
   \begin{tabular}{c |l| c} 
   \noalign{\smallskip}\hline\noalign{\smallskip}
   {\bf Component field} & {\bf Type} & {\bf d.o.f.} \\
   \noalign{\smallskip}\hline\noalign{\smallskip}
   $C(x)$, $D(x)$ & Real scalar field& 1\\
  $\phi_A(x)$, $\lambda_A(x)$ &Weyl spinor & 4\\
  $M(x)$ &Complex scalar field & 2\\
  $V_\mu(x)$ & Real Lorentz 4-vector & 4\\
  \noalign{\smallskip}\hline\noalign{\smallskip}
    \end{tabular}
   \captionof{table}{Field content of a general vector superfield. \label{tab:vecsup}}
   \end{center}

With the large number of component fields, and their strange mass dimensions, you may now be a little suspicious that this vector superfield does not correspond to the promised degrees of freedom in the $j=\frac{1}{2}$ representation of the superalgebra, even after the application of the equations of motion. However, gauge-freedom now comes to our rescue.



%%%%%%%%%%%
\section{Supergauge}
%%%%%%%%%%%
We first define what we will mean by an abelian (super)gauge transformation of a superfield.\footnote{We promise that we will get back to the corresponding definition for non-abelian transformations.} Later we will see how it relates to the ordinary gauge transformations of quantum fields. We begin with the scalar superfields.

\df{The abelian {\bf supergauge transformation} (local or global) on a left handed scalar superfield $\Phi_i$ is defined as:
\begin{equation}
\Phi_i \to \Phi'_i = e^{iq_i\Lambda}\Phi_i
\label{eq:scalar_supergauge}
\end{equation}
where $q_i$ is the charge of $\Phi_i$ under that gauge group and $\Lambda$ (global), or $\Lambda(x)$ (local), is the parameter of the gauge transformation.}
The definition is of course completely equivalent for right-handed scalar fields and generalises the standard definition of (abelian) gauge transformation in quantum field theory.
For the definition to make sense the transformed field $\Phi_i'$ must be a left-handed scalar superfield itself, thus
\[\bar{D}_{\dot{A}}\Phi'_i = 0,\]
and this requires:
\begin{eqnarray*}
\bar{D}_{\dot{A}}\Phi'_i = \bar{D}_{\dot{A}}e^{iq_i\Lambda }\Phi_i =  e^{iq_i\Lambda }\bar{D}_{\dot{A}}\Phi_i +iq_i(\bar{D}_{\dot{A}}\Lambda)e^{i q_i\Lambda}\Phi_i
= iq_i(\bar{D}_{\dot{A}}\Lambda)\Phi_i'=0.
\end{eqnarray*}
Thus we must have $\bar{D}_{\dot{A}}\Lambda = 0$, which means that  the parameter $\Lambda$ is also a left-handed superfield. Note, however, that $\Lambda$ has must have a mass dimension $[\Lambda]=M^0$ in order for the exponentiation to make sense.

Next, we move to the vector superfields.
\df{Given a vector superfield $V(x,\theta,\bar{\theta})$, we define the {\bf abelian supergauge transformation} as
\begin{eqnarray}
V(x,\theta,\bar{\theta}) \to V'(x,\theta,\bar{\theta}) &=& V(x,\theta,\bar{\theta}) - i[ \Lambda(x, \theta, \bar{\theta}) -  \Lambda^\dagger(x, \theta, \bar{\theta})],
\label{eq:abelian_supergauge_vector_superfield}
\end{eqnarray}
where the parameter of the transformation $\Lambda$ is a scalar superfield.}


With the expressions for scalar superfields in (\ref{eq:leftscalarsuperfield}) and (\ref{eq:rightscalarsuperfield}), using $\Phi=i\Lambda$, we can show that under supergauge transformations the vector superfield components transform as:
\begin{eqnarray}
C(x) &\to& C'(x) = C(x) + A(x) + A^*(x)\\
\phi(x) &\to& \phi'(x) = \phi(x) + \sqrt{2}\psi(x)\\
M(x) &\to& M'(x) = M(x) + F(x)\\
V_\mu(x) &\to& V_\mu'(x) = V_\mu(x) +i\partial_\mu(A(x) - A^*(x))\\
\lambda(x) &\to& \lambda'(x) = \lambda(x)\\
D(x) &\to& D'(x) = D(x). \label{eq:supergaugetrans_Dterm}
\end{eqnarray}
If we look at the transformation this implies for the vector field, this is equivalent to the ordinary abelian gauge transformation for a vector field  $V_\mu(x)\to V_\mu'(x) = V_\mu(x) +\partial_\mu g(x)$, with the gauge parameter given by the scalar component field of $\Lambda$, $g(x)=i[A(x) - A^*(x)]=-2\Im{A(x)}$. It then immediately follows that the standard field strength for a vector field, $F_\mu{}_\nu = \partial_\mu V_\nu - \partial_\nu V_\mu$, is supergauge invariant. 

If we demand that our theory is invariant under these gauge transformations we can choose the component fields of $\Lambda$ in order to eliminate some the remaining reducibility  in the representation. 
\df{The {\bf Wess-Zumiono (WZ) gauge} is a supergauge transformation of a vector superfield by a scalar superfield with
\begin{eqnarray}
\psi(x) &=& -\frac{1}{\sqrt{2}}\phi(x),\\
F(x) &=& -M(x),\\
A(x) + A^*(x) &=& 2\Re{A(x)} = -C(x).
\end{eqnarray}}
A vector superfield in the WZ-gauge can then be written:
\begin{equation}
V_{WZ} (x, \theta, \bar{\theta}) = (\theta\sigma^\mu \bar{\theta})[V_\mu(x) + i\partial_\mu(A(x) - A^*(x))] + \theta\theta\bar{\theta}\bar{\lambda}(x) + \bar{\theta}\bar{\theta}\theta\lambda (x) + \theta\theta\bar{\theta}\bar{\theta}D(x).
\label{eq:vectorsuperfieldWZ}
\end{equation}
Again, the equations of motion will eliminate the auxiliary $D$-field, as well as one d.o.f.\ from the gauge field leaving the  two d.o.f.\ of a massless on-shell gauge boson as found in the Standard Model,\footnote{Hang on, where did that last d.o.f.\ go from $V(x)$? We have a remaining gauge freedom in the choice of the imaginary component of $A(x)$, which is the ordinary gauge freedom of a $U(1)$ field theory. This can be used to eliminate one d.o.f.\ from the vector field.}  and two d.o.f.\ from the (Majorana) fermion formed by $\lambda$, usually called the {\bf gaugino} partner of the gauge boson. This does contain the correct number of degrees of freedom that corresponds to the representation $j=\frac{1}{2}$, however, for the massless $m=0$ representation.\footnote{The vector bosons will get mass through electroweak symmetry breaking just as in the Standard Model.}

Notice that the WZ gauge is particularly convenient for calculations because:
\begin{equation}
V_{WZ}^2 = \frac{1}{2}\theta\theta\bar{\theta}\bar{\theta}[V_\mu(x) + i\partial_\mu(A(x) - A^*(x))][V^\mu(x) + i\partial^\mu(A(x) - A^*(x))],
\label{eq:V2_WZ}
\end{equation}
and, since multiplying in any $\theta$ or $\bar\theta$ into $V_{WZ}^2$ will then yield zero, we have 
\[V_{WZ}^3 = 0,\]
so that 
\[e^{V_{WZ}} = 1 + V_{WZ} + \frac{1}{2}V_{WZ}^2.\]
Unfortunately, supersymmetry transformations break the Wess-Zumiono gauge, meaning that a vector superfield in the WZ-gauge will no longer be in the WZ-gauge after a supersymmetry transformation. 



%%%%%%%%%
\section{Exercises}
%%%%%%%%%
\noindent

\begin{Exercise}
Show that
\[\int d^2\theta\,\theta\theta = 1.\]
\end{Exercise}

\begin{Answer}
\begin{eqnarray*}
\int d^2\theta\,\theta\theta &=& - \frac{1}{4} \int d\theta^Ad\theta_A\,\theta^A\theta_A \\
&=& \frac{1}{2} \int d\theta^Ad\theta_A\,\theta_1\theta_2
= - \int d\theta_1d\theta_2\,\theta_1\theta_2=\int d\theta_1\theta_1\int d\theta_2\theta_2=1
\end{eqnarray*}
\end{Answer}

\begin{Exercise}
Check that the explicit differential forms of the generators in Eqs.~(\ref{eq:diffrepP})--(\ref{eq:diffrepQbar}) fulfil the superalgebra in Eqs.~(\ref{eq:QQweyl})--(\ref{eq:QPweyl}).
\end{Exercise}

\begin{Exercise}
Demonstrate the correctness of the general expression for the left-handed scalar superfield in Eq.~(\ref{eq:leftscalarsuperfield}). {\it Hint:} You may have use for the spinor identities
\[ (\theta\sigma^\mu \bar{\theta})(\theta\sigma^\nu \bar{\theta})=\frac{1}{2}g^{\mu\nu}\theta\theta\bar\theta\bar\theta, \]
\[ \theta\partial_\mu\psi\theta\sigma^\mu \bar{\theta}=\frac{1}{2}\theta\theta\partial_\mu \psi\sigma^\mu\bar{\theta}. \]
\end{Exercise}

\begin{Answer}
We start from
\begin{equation*}
\Phi(y, \theta) = A(y) + \sqrt{2}\theta\psi(y) + \theta\theta F(y),
\end{equation*}
and $y^\mu \equiv x^\mu + i\theta\sigma^\mu \bar{\theta}$. This gives
\begin{eqnarray*}
\Phi(x, \theta, \bar\theta) &=& A(x + i\theta\sigma\bar{\theta}) + \sqrt{2}\theta\psi(x + i\theta\sigma\bar{\theta}) + \theta\theta F(x + i\theta\sigma \bar{\theta}) \\
&=& A(x) + \partial_\mu A(x)i\theta\sigma^\mu \bar{\theta} + \frac{1}{2}\partial_\mu\partial_\nu A(x)(i\theta\sigma^\mu \bar{\theta})(i\theta\sigma^\nu \bar{\theta})+ \sqrt{2}\theta\psi(x) \\
&&+ \sqrt{2}\partial_\mu\theta\psi(x)i\theta\sigma^\mu \bar{\theta} + \theta\theta F(x) \\
&=& A(x) + i\theta\sigma^\mu \bar{\theta}\partial_\mu A(x) - \frac{1}{2}\partial_\mu\partial_\nu A(x)\frac{1}{2}g^{\mu\nu}\theta\theta\bar\theta\bar\theta+ \sqrt{2}\theta\psi(x) \\
&& + i\sqrt{2}\theta\partial_\mu\psi(x)\theta\sigma^\mu \bar{\theta} + \theta\theta F(x) \\
&=& A(x) + i\theta\sigma^\mu \bar{\theta}\partial_\mu A(x) - \frac{1}{4} \Box A(x)\theta\theta\bar\theta\bar\theta+ \sqrt{2}\theta\psi(x)+\frac{i}{\sqrt{2}}\theta\theta\partial_\mu \psi\sigma^\mu\bar{\theta} + \theta\theta F(x),
\end{eqnarray*}
where we have used that $(\theta\sigma^\mu \bar{\theta})(\theta\sigma^\nu \bar{\theta})=\frac{1}{2}g^{\mu\nu}\theta\theta\bar\theta\bar\theta$ and 
$\theta\partial_\mu\psi\theta\sigma^\mu \bar{\theta}=\frac{1}{2}\theta\theta\partial_\mu \psi\sigma^\mu\bar{\theta}$.
\end{Answer}

\begin{Exercise}
Show the vector superfield supergauge transformation properties for the component fields. {\it Hint:} Use the field redefinitions:
\[\lambda(x) \to \lambda(x) -\frac{i}{2}\sigma^\mu\partial_\mu\bar{\phi}(x),\]
\[D(x) \to D(x) - \frac{1}{4}\Box C(x).\]
\end{Exercise}

\begin{Exercise}
Derive the expression for $V_{WZ}^2$ in Eq.~(\ref{eq:V2_WZ}).
\end{Exercise}

%\begin{Exercise}
%We start from the original superfield in (\ref{eq:leftscalarsuperfield})
%\begin{equation}
%\Phi(x, \theta, \bar{\theta}) = A(x) + i(\theta\sigma^\mu \bar{\theta})\partial_\mu A(x) - \frac{1}{4}\theta\theta\bar{\theta}\bar{\theta}\Box A(x) + \sqrt{2}\theta \psi(x) - \frac{i}{\sqrt{2}}\theta\theta\partial_\mu \psi(x)\sigma^\mu\bar{\theta} + \theta\theta F(x).
%\end{equation}
%
%The superfield transformation parameter is
%\begin{equation}
%i\Lambda= A'(x) + i(\theta\sigma^\mu \bar{\theta})\partial_\mu A'(x) - \frac{1}{4}\theta\theta\bar{\theta}\bar{\theta}\Box A'(x) + \sqrt{2}\theta \psi'(x) - \frac{i}{\sqrt{2}}\theta\theta\partial_\mu \psi'(x)\sigma^\mu\bar{\theta} + \theta\theta F'(x).
%\end{equation}
%\begin{equation}
%e^{i\Lambda}= e^{A'(x)}[1+ 2i(\theta\sigma^\mu \bar{\theta})\partial_\mu A'(x) - \frac{1}{2}\theta\theta\bar{\theta}\bar{\theta}\Box A'(x) + 2\sqrt{2}\theta \psi'(x) - 2\frac{i}{\sqrt{2}}\theta\theta\partial_\mu \psi'(x)\sigma^\mu\bar{\theta} + 2\theta\theta F'(x)
%-\frac{1}{2}\theta\theta \bar{\theta}\bar{\theta}\partial_\mu A'(x) \partial^\mu A'(x) 
%+\sqrt{2} i(\theta\sigma^\mu \bar{\theta})\partial_\mu A'(x)\theta \psi'(x) 
%+ 2\theta \psi'(x)\theta \psi'(x)].
%\end{equation}
%
%\begin{equation}
%e^{i\Lambda}= e^{A'(x)}[1+ 2i(\theta\sigma^\mu \bar{\theta})\partial_\mu A'(x) - \frac{1}{2}\theta\theta\bar{\theta}\bar{\theta}\Box A'(x) + 2\sqrt{2}\theta \psi'(x) - 2\frac{i}{\sqrt{2}}\theta\theta\partial_\mu \psi'(x)\sigma^\mu\bar{\theta} + 2\theta\theta F'(x)
%-\frac{1}{2}\theta\theta \bar{\theta}\bar{\theta}\partial_\mu A'(x) \partial^\mu A'(x) 
%+\sqrt{2} i(\theta\sigma^\mu \bar{\theta})\partial_\mu A'(x)\theta \psi'(x) 
%+ 2\theta \psi'(x)\theta \psi'(x)].
%\end{equation}
%
%\end{Exercise}


\end{document}


% Standalone document
\documentclass[notes.tex]{subfiles}
\begin{document}
%%%%%%%%%%%%%%%%%%%%%%%%%%%%%%%%%%%%%%%%%%%%%%%%%%%%%%%
%%%%%%%%%%%%%%%%%%%%%%%%%%%%%%%%%%%%%%%%%%%%%%%%%%%%%%%
\chapter{The Minimal Supersymmetric Standard Model}
\label{chap:mssm}
%%%%%%%%%%%%%%%%%%%%%%%%%%%%%%%%%%%%%%%%%%%%%%%%%%%%%%%
%%%%%%%%%%%%%%%%%%%%%%%%%%%%%%%%%%%%%%%%%%%%%%%%%%%%%%%
The Minimal Supersymmetric Standard Model (MSSM) is a minimal supersymmetric model in the sense that it has the smallest field (and gauge) content consistent with the known Standard Model fields. We will now construct this model on the basis of what we have learnt the previous chapters, and look at some of its consequences.


%%%%%%%%%%%%%%%
\section{MSSM field content}
%%%%%%%%%%%%%%%
To specify a supersymmetric model we need to specify the superfield content and the gauge symmetries. The gauge symmetry is already given as the Standard Model $SU(3)_c\times SU(2)_L\times U(1)_Y$. We start now with writing down the fields we need for the Standard Model fermions. Previously we learnt that each (left-handed) scalar superfield S has a (left-handed) Weyl spinor $\phi_A$, a complex scalar $s$ and an auxiliary complex scalar field $F$, since they are a $j=0$ representation of the superalgebra. After an application of the equations of motion $\phi_A$ and $s$ have two fermionic and two bosonic degree of freedom remaining respectively, while the auxiliary field has been eliminated along with two fermionic degrees of freedom. 

In order to construct a Dirac fermion, which are plentiful in the Standard Model, we need a right-handed Weyl spinor as well. We can acquire the needed right-handed Weyl spinor from the hermitian conjugate $\bar{T}^\dagger$ of a different scalar superfield $\bar{T}$ with the right-handed Weyl spinor $\bar\chi_{\dot{A}}$ and the complex scalar $t^*$.\footnote{The bar here is used to (not) confuse us, it is part of the name of the superfields and does not denote any hermitian or complex conjugate. The bar signifies that $\bar T$ is the field where, when hermitian conjugated into $\bar{T}^\dagger$, we will pick the right-handed Weyl-spinor to use in the Dirac fermion, while the left-handed Weyl spinor in the bared field $\bar T$  itself belongs to the corresponding anti-particle. Since $SU(2)_L$ acts only on the left-handed Weyl spinors of particles (as opposed to anti-particles), another way to think about this is that the left-handed Weyl-spinor in the bared field  $\bar T$ is the one that does not transform under $SU(2)_L$.} With these four fermionic d.o.f.\ we can construct {\it two} Dirac fermions, a particle--anti-particle pair,
\[ \psi_a= \left[ \begin{matrix}\phi_A \\ \bar\chi^{\dot{A}} \end{matrix} \right], \quad  \bar\psi_a= \left[ \chi^A ,\, \bar\phi_{\dot{A}}  \right], \]
and four scalars, two particle--anti-particle pairs, $s$, $s^*$, $t$ and $t^*$.

We use these two superfield ingredients to construct all the known fermions:
\begin{itemize}
\item To represent the Standard Model leptons we introduce the superfields $l_i$ and $\bar{E}_i$ for the charged leptons ($i$ is the generation index) and $\nu_i$ for the neutrinos, and form $SU(2)_L$ doublet superfield vectors $L_i = (\nu_i, l_i)$, which contain the left-handed Weyl spinors for the particles, while $\bar{E}_i$  contain the left-handed Weyl spinors that are singlets (do not transform) under $SU(2)_L$.
\item There is an imbalance in the neutrinos in that we do not introduce the neutrino superfield $\bar{N}_i$ that through $\bar{N}_i^\dagger$ would contain right-handed neutrino spinors needed for massive Dirac neutrinos, instead leaving the neutrinos as massless Majorana particles in the MSSM.\footnote{The anti-neutrino contained in the superfield $\nu_i^\dagger$ is a right-handed Weyl-spinor consistent with experiment.}
This is a convention -- the MSSM is older than the discovery of neutrino mass -- and including $\bar{N}_i$ fields would have some interesting consequences.
%\footnote{Note that component fields in the same superfield must have the same charge under all the gauge groups, {\it i.e.}\ the scalar partner of the electron has electric charge $-e$, so it cannot be a neutrino.}
The $\bar{N}_i$ superfields and their component fields would not couple to any of the gauge fields since they are SM singlets,\footnote{They can not be colour-charged, they are singlets under $SU(2)_L$ by construction thus they have zero weak isospin $I_3$, but since they should also have zero electric charge $Q$, the hypercharge $Y$ must also be zero through the relationship $Q=\half Y+I_3$.} however, the scalar field in the superfield could be a potential dark matter candidate.
\item For quarks the situation is similar. Up-type and down-type quarks get the superfields $u_i$ and $d_i$,  forming the $SU(2)_L$ doublets $Q_i = (u_i, d_i)$, and the $SU(2)_L$  singlets $\bar{U}_i$ and $\bar{D}_i$.\footnote{Here we should really also include a colour index $a$ so that $u_i^a$ is a component in an $SU(3)_c$ triplet superfield vector. We omit these for simplicity.}
\end{itemize}

With all possible apologies, we now change notation for the component fields to what is closer to conventions in phenomenology, as opposed to pure theory. The left-handed Weyl spinors in these superfields will now be named for example $l_i$ for the spinor in the superfield $l_i$, $\nu_i$ for the one in the $\nu_i$ superfield, and $\bar e_{i}$ for the one in $\bar E_i$,  $u_{i}$ for the one in the $u_i$ superfield, $d_i$ for the one in the $d_i$ superfield, $\bar u_i$ for the one in the $\bar U_i$ superfield,  and finally $\bar d_i$ for the one in the $\bar D_i$ superfield. This means that we leave our former notation where the bar signifies right-handed Weyl spinor, but now instead signifies the left-handed Weyl spinor for the anti-particle, or, in other words, a $SU(2)_L$ singlet. If we now want to indicate a right-handed Weyl spinor we use the hermitian conjugation in (\ref{eq:Weyl_hc}) as an alternative notation, so that for example the $l_i^\dagger$ superfield contains $\bar l_i^\dagger$ and $\bar E_i^\dagger$ contains $\bar e_i^\dagger$, and so on. The reason for this change is in part that we need a simple way to write down component field Lagrangians with fermions without greek letters and very many indices indicating particle type, as well as running out of letters since we need two superfields for one fermion.
% An alternative notation would be to write $e_{Li}$ for the left-handed Weyl-spinor in $l_i$ and $e_{Ri}$ for the left-handed Weyl-spinor in $\bar E_i$ etc.
% TODO: Work out the Weyl -> Dirac transition and re-consider the notation

The scalar component fields are named after the fermions using the tilde notation, for example in the superfield $l_i$ we have the scalar $\tilde l_{iL}$ as the supersymmetric partner particle, often just called {\bf sparticle}, of the fermion $l_i$.  Similarly, $\bar E_i$ contains $\tilde l_{iR}^*$, $l_i^\dagger$ contains $\tilde l_{iL}^*$, and $\bar E_i^\dagger$ contains $\tilde l_{iR}$.  Since scalars do not have any notion of handedness the $L$ or $R$ here is just part of the conventional name; we still call these particles for left-handed and right-handed  scalar leptons though.  The complex conjugates might be surprising, but remember that for example the superfield $\bar E_i$ contains the left-handed Weyl-spinor of the {\it anti}-leptons and thus has positive electric charge. The collective term for these scalars are {\bf sfermions}.

Additionally, we need vector superfields, which, after the equations of motion  have eliminated the auxiliary field, contain a {\it massless} vector boson $V_\mu$ component field with two scalar degrees of freedom and a Weyl-spinor $\lambda_A$, with two fermionic degrees of freedom. Together these form a $m=0$, $j=\frac{1}{2}$ representation of the superalgebra. If the vector superfield is neutral, the Weyl-spinor  can form a Majorana fermion, if not it can be combined with the Weyl-spinor from another vector superfield to form a Dirac fermion. 

Looking at the construction $V\equiv qT_aV^a$ for the vector superfields in the supersymmetric Lagrangian we see that, as expected, we need one superfield $V^a$ per generator $T_a$ of the algebra, with the normal $SU(3)_c$, $SU(2)_L$ and $U(1)_Y$ vector bosons as vector component fields. We call these superfields $C^a$, $W^i$ and $B^0$, where $a=1,\ldots,8$ and $i=1,2,3$.\footnote{And there we have another W.} 
In order to be really confusing, we use the following symbols for the fermion Weyl-spinors: $\tilde g^a$, $\tilde W^+$, $\tilde W^0$, $\tilde W^-$ and $\tilde B^0$. The tilde here is supposed to tells us -- just as for the scalar component fields of the scalar superfields -- that they are supersymmetric partners of the known Standard Model particles. These particles are collectively known as the {\bf gauginos}. 

We also need superfields for the scalar Higgs boson. Now life gets interesting. The Higgs $SU(2)_L$ doublet scalar field $H$ used in the Standard Model cannot give mass to all fermions because it relies on the construction $H^C \equiv -i(H^\dagger\sigma_2)^T$ to give masses to up-type quarks, and possibly neutrinos. The superfield version of this cannot appear in the superpotential because it would mix left- and right-handed superfields due to the hermitian conjugation in $H^C$. The minimal Higgs content we can get away with are two Higgs superfield $SU(2)_L$ doublets, which we will call $H_u$ and $H_d$, indexing the quarks they give mass to.\footnote{In some further insanity some authors prefer $H_1$ and $H_2$ so that you have no idea which is which.} These must have (more on the reason for that in Sec.~\ref{sec:mssm_kinetic_terms}) weak hypercharge $Y = \pm 1$ for $H_u$ and $H_d$, respectively, so that we have the superfield doublets:
\begin{equation}
H_u = \begin{pmatrix} H_u^+\\H_u^0\end{pmatrix},\quad
H_d = \begin{pmatrix} H_d^0\\H_d^-\end{pmatrix},
\end{equation}
where we have given the electric charges of the scalar superfield components of the superfield doublets based on the standard $Q=\half Y+T_3$ relationship after electroweak symmetry breaking, where $T_3$ is the weak isospin.\footnote{The upper component of a doublet has $T_3=\half$ while the lower has $T_3=-\half$. This is again just the eigenvalues of the $J_3$ generator in two-dimensional representation of the $SU(2)$ group, see Sec.~\ref{sec:SU2_irreps}.} The scalar component fields of these fields, before their mixing following electroweak symmetry breaking, will have the same symbols as the superfields. (Yes, really!)
The fermion component fields will be denoted $\tilde H_u^+$, $\tilde H_u^0$, $\tilde H_d^0$, and $\tilde H_d^-$, and are known as the {\bf higgsinos}.


%%%%%%%%%%%%%
\section{The kinetic terms}
\label{sec:mssm_kinetic_terms}
%%%%%%%%%%%%%
It is now straight forward to write down the kinetic terms of the MSSM Lagrangian giving the matter-gauge interaction terms
\begin{eqnarray}\label{eq:kinlag}
\mathcal{L}_\text{kin} &=& L_i^\dagger e^{\frac{1}{2}g \sigma W - \frac{1}{2} g' B}L_i + Q_i^\dagger e^{\frac{1}{2}g_s\lambda C + \frac{1}{2}g\sigma W + \frac{1}{3}\cdot \frac{1}{2} g' B}Q_i \nonumber\\
&&+\bar{U}_{i}^\dagger e^{\frac{1}{2}g_s \lambda C - \frac{4}{3}\cdot \frac{1}{2} g' B}\bar{U}_i + \bar{D}_i^\dagger e^{\frac{1}{2} g_s\lambda C + \frac{2}{3}\cdot \frac{1}{2} g' B}\bar{D}_i \nonumber\\
&& + \bar{E}_i^\dagger e^{2\frac{1}{2} g' B}\bar{E}_i + H_u^\dagger e^{\frac{1}{2} g \sigma W + \frac{1}{2} g'B}H_u + H_d^\dagger e^{\frac{1}{2}g \sigma W - \frac{1}{2} g' B}H_d,
\end{eqnarray}
where $g'$, $g$ and $g_s$ are the couplings constants (strengths) of $U(1)_Y$, $SU(2)_L$, and $SU(3)_c$, respectively, and $\half\sigma_i$ and $\half\lambda_a$ are the generators of  $SU(2)_L$ and $SU(3)_c$. As a convention we assign the charges under $U(1)$, hypercharge, in units of $\frac{1}{2}g'$. For a list of the hypercharge assignments see Table~\ref{tab:hyper}.  All non-singlets of $SU(2)_L$ and $SU(3)_C$ have the same charge, the factor $\frac{1}{2}$ here is used to get by without accumulation of numerical factors since the algebras for the Pauli ($\sigma_i$) and Gell-Mann matrices ($\lambda_a$) are:
\[ \left[\frac{1}{2}\sigma_i, \frac{1}{2} \sigma_j\right] = i \epsilon_{ijk}\frac{1}{2}\sigma_k,\]
and
\[ \left[\frac{1}{2}\lambda_a, \frac{1}{2} \lambda_b\right] = i f_{ab}^{~~c}\frac{1}{2}\lambda_c.\]
These conventions lead to the wanted Standard Model gauge transformations for the component fields and the familiar relations after electroweak symmetry breaking, $e = g\sin\theta_W = g'\cos\theta_W$, where $e$ is the elementary electric charge (in natural units).

We mentioned earlier that the two Higgs superfields have opposite hypercharge. This is needed for so-called {\bf anomaly cancellation} in the MSSM. Gauge anomaly is the possibility that at loop level contributions to processes such as in Fig.~\ref{fig:anomaly} break the gauge invariance that we have established at the classical level in the Lagrangian, and ruins the predictability of the theory. This rather miraculously does not happen in the SM because it has the exactly field content it has, so that all such possible gauge anomalies exactly cancel -- we do not know of a deeper reason for why it has exactly this field content. If we have only {\it one} Higgs doublet this cancellation does not happen for the MSSM. With two Higgs doublets, and with opposite hypercharge, it does.

% TODO: introduce more details here, including hypercharge counting (Sandvik)

%%%
\begin{figure}[h]
\centering
\begin{tikzpicture}
\begin{feynman}
%
\diagram [layered layout, horizontal=a to b] {
a [particle=$B$] -- [boson] b,
b -- [boson] c [particle=$B$],
b -- [boson] d [particle=$B$],
};
%
\diagram [xshift=6cm, small, horizontal=a to t1] {
a [particle=$B$] -- [boson] t1 -- [fermion] t2 -- [fermion, edge label=$f$] t3 -- [fermion] t1,
t2 -- [boson] p1 [particle=$B$],
t3 -- [boson] p2 [particle=$B$],
p1 -- [opacity=0] p2,
};
\end{feynman}
\end{tikzpicture}
\caption{The tree level coupling between three gauge bosons $B$ (left), and the one-loop fermion contribution to the same process (right).}  
\label{fig:anomaly}
\end{figure}
%%%



%%%%%%%%%%%
\section{Gauge terms}
%%%%%%%%%%%
The pure gauge terms with supersymmetric field strengths are also fairly easy to write down:
\begin{eqnarray}\label{eq:gaugelag}
\mathcal{L}_V = \frac{1}{2g_s^2}\bar{\theta}\bar{\theta}\tr{C^AC_A} +\frac{1}{2g^2}\bar{\theta}\bar{\theta}\tr{W^AW_A} +  \frac{1}{2g'^2}\bar{\theta}\bar{\theta}B^AB_A+\text{h.c.},
\end{eqnarray}
where we have used the Dynkin indices of the gauge group representations
\[T(R)_L = {\rm Tr\,}\left[\frac{1}{2}\sigma_1\cdot \frac{1}{2}\sigma_1\right] = \frac{1}{2},\]
and
\[T(R)_c = {\rm Tr\,}\left[\frac{1}{2}\lambda_1\cdot \frac{1}{2}\lambda_1\right] = \frac{1}{2},\]
in the normalization of the terms, and where the field strengths are given as:
\begin{eqnarray}
C_A &=& -\frac{1}{4}\bar{D}\bar{D}e^{-C}D_Ae^C \text{ , \indent \indent}C = \frac{1}{2}g_s \lambda_aC^a,\\
W_A &=& -\frac{1}{4}\bar{D}\bar{D}e^{-W}D_Ae^W \text{ , \indent \indent}W = \frac{1}{2} g \sigma_iW^i,\\
B_A &=& -\frac{1}{4}\bar{D}\bar{D}D_AB \text{ , \indent \indent}B = \frac{1}{2} g' B^0.
\end{eqnarray}



%%%%%%%%%%%%%%%%%%
\section{The MSSM superpotential}
%%%%%%%%%%%%%%%%%%
With the gauge structure of the Standard Model  in place we are ready to write down all possible terms in the superpotential. First, we notice that there can be no {\bf tadpole terms} $t_i\Phi_i$ (terms with only one superfield), since there are no superfields that are singlets (zero charge) under all Standard Model gauge groups. The only alternative would be if we introduced right-handed neutrino superfields $\bar{N}_i$. 

For the possible {\bf mass terms} $m_{ij}\Phi_i\Phi_j$ we first check the abelian gauge group $U(1)_Y$, where the requirement reduces to the relatively simple sum of hypercharges $Y_i + Y_j = 0$. In Table~\ref{tab:hyper} we see that the only contributions  with sum zero hypercharge using  superfields that contain the Standard Model fermions are particle--anti-particle combinations such as $l_i l_i^\dagger$, but these come from superfields with different handedness ($L_i$ and $L_i^\dagger$) and cannot be used together. Thus the superpotential {\it can not} be used to give masses to the Standard Model fermions, and we will need a Higgs mechanism in the MSSM as well.

\begin{table}[h]
\begin{center}
\begin{tabular}{l | c | c | c | c | c | c | c  } 
\noalign{\smallskip}\hline\noalign{\smallskip}
{\bf Superfield} & $L_i$ & $\bar{E}_i$ & $Q_i$ & $\bar{U}_i$ & $\bar{D}_i$ & $H_u$ & $H_d$  \\
\noalign{\smallskip}\hline\noalign{\smallskip} 
%{\bf Fermion}    & $\nu_i{}_L$, $l_i{}_L$ & $l_i{}_R$ & $u_i{}_L$,$d_i{}_L$ & $u_i{}_R$ & $d_i{}_R$ & $\bar{\nu}_i{}_R$, $\bar{l}_i{}_R$ & $\bar{l}_i{}_L$ \\
{\bf Fermion}    & $\nu_i$, $l_i$ & $\bar e_i$ & $u_i$, $d_i$ & $\bar u_i$ & $\bar d_i$ & $\tilde H_u^+$, $\tilde H_u^0$ & $\tilde H_d^0$, $\tilde H_d^-$ \\
{\bf Hypercharge} $Y$ & $-1$ & $2$ & $\frac{1}{3}$ & $-\frac{4}{3}$ & $\frac{2}{3}$ & $1$ & $-1$ \\
{\bf Electric charge} $Q$ & $0$, $-1$ & $1$ & $\frac{2}{3}$, $-\frac{1}{3}$ & $-\frac{2}{3}$ & $\frac{1}{3}$ & $1$, $0$ & $0$, $-1$ \\
\noalign{\smallskip}\hline\noalign{\smallskip}
\end{tabular}
\caption{The MSSM superfields with their Standard Model fermion content, hypercharge $Y$, and electric charge $Q$.}
\label{tab:hyper}
\end{center}
\end{table}

Going beyond the superfields with Standard Model fermions we see that we can make a mass term with the two Higgs superfields that have opposite hypercharge $Y=\pm1$. These fields are not charged under $SU(3)_c$, but in order to also be invariant under $SU(2)_L$ we have to write this superpotential term as
\begin{equation}
W_\text{mass}=\mu H^T_ui\sigma_2 H_d,
\label{eq:higgsmasstermsuperpot}
\end{equation}
where $\mu\in\mathbb C$ is the superpotential mass parameter for this term. This construction is invariant under $SU(2)_L$ because, with the supergauge transformations $H_d \to e^{ig\frac{1}{2}\sigma_kW^k}H_d$ and $H^T_u \to H^T_u e^{ig\frac{1}{2}\sigma^T_kW^k}$, we get
\begin{eqnarray*}
H_u^T i\sigma_2 H_d &\to& H_u^T e^{ig\frac{1}{2}\sigma_k^TW^k}i\sigma_2 e^{ig\frac{1}{2}\sigma_k W^k}H_d\\
 &=& H_u^T i\sigma_2e^{-i\frac{1}{2}g\sigma_kW^k}e^{i\frac{1}{2}g\sigma_kW^k}H_d = H_u^T i\sigma_2 H_d,
\end{eqnarray*}
since $\sigma_k^T\sigma_2 = -\sigma_2\sigma_k$. Usually we ignore the $SU(2)_L$ specific structure and write terms like  this as $\mu H_u H_d$, confusing the hell out of anyone that is not used to this convention since we really do mean Eq.~(\ref{eq:higgsmasstermsuperpot}). Notice that if we write (\ref{eq:higgsmasstermsuperpot}) in terms of the component superfields in the two $SU(2)_L$ doublets we get
\[W_\text{mass}=\mu H_u^Ti\sigma_2 H_d = \mu (H_u^+ H_d^- - H_u^0 H_d^0),\]
which we should have been able to guess because the Lagrangian must also conserve electric charge.

If you have paid very close attention to the argument above you may have noticed that there is one more possibility, namely
\[W_\text{mass}=\mu'_i L_iH_u \equiv \mu_i' L_i^T i\sigma_2 H_u = \mu_i'(\nu_i H_u^0 - l_iH_u^+),\]
where $\mu'\in\mathbb C$ is some other mass parameter in the superpotential. This is clearly an allowable term (and we will return to it below), however, it also raises a very interesting question: Could we have $L_i\equiv H_d$? Could the lepton superfields $L_i$ play the r\^ole of Higgs superfields, thus reducing the field content needed to describe the SM particles in a supersymmetric theory? While not immediately forbidden as a superpotential term, this suggestions unfortunately leads to problems with anomaly cancelation, processes with large lepton flavour violation (LFV) and much too massive neutrinos, and has been abandoned.

We have now found all possible mass terms in the superpotential. What about the {\bf Yukawa terms} $\lambda_{ijk}\Phi_i\Phi_j\Phi_k$? The hypercharge requirement here is $Y_i + Y_j + Y_k = 0$. From our table of hypercharges only the following terms are found to be viable:
\[W_\text{Yukawa}=y^e_{ij}L_iH_d E_j + y^u_{ij}Q_iH_u\bar{U}_j +  y^d_{ij}Q_iH_d\bar{D}_j+ \lambda_{ijk} L_iL_j\bar{E}_k + \lambda'_{ijk} L_iQ_j\bar{D}_k + \lambda''_{ijk}\bar{U}_i\bar{D}_j\bar{D}_k,\]
where we have named and indexed the Yukawa couplings in a hopefully natural way.\footnote{For some particular opinion of what is natural.} 
For all these terms we can simultaneously keep $SU(2)_L$ invariance with the $i\sigma_2$ construction implicitly inserted between any two superfield doublets.  Note that because of the $SU(2)_L$  invariance, we must have $i\ne j$ for  $\lambda_{ijk}$, since $i=j$ gives $ L_iL_i\bar{E}_k =(\nu_i l_i- l_i\nu_i)\bar E_k=0$.

For $SU(3)_c$ to be conserved for the Yukawa terms, we need to have colour singlets. Some of these terms are colour singlets by construction since they do not contain any coloured fields -- the $LHE$ and $LLE$ terms. The terms with only two quark superfields contain left-handed Weyl spinors for quarks and anti-quarks, which form $SU(3)_c$ singlets if the superfields come in colour--anti-colour pairs. In representation language the superfields (and as a consequence their component fields) are in the $\mathbf 3$ and $\mathbf{\bar 3}$ representations of $SU(3)_c$. Written with explicit  colour indices we have for example $L_i Q_j \bar{D}_k= L_i i\sigma_2 Q_j^a \bar{D}_k^a$, where $a$ is the colour index. The final term $\bar{U}_i\bar{D}_j\bar{D}_k$ is a colour singlet once we demand that it is totally anti-symmetric in the colour indices: $\bar{U}_i\bar{D}_j\bar{D}_k\equiv\epsilon_{abc}\bar{U}_i^a\bar{D}_j^b\bar{D}_k^c$. The anti-symmetry property of the Levi-Civita tensor $\epsilon_{abc}$ means that we must have $j\ne k$ in $ \lambda''_{ijk}$.

Our complete superpotential is then:
\begin{eqnarray}
\label{eq:supolag}
W &=& \mu H_u H_d + \mu'_i L_iH_u + y^e_{ij}L_iH_d E_j + y^u_{ij}Q_iH_u\bar{U}_j +  y^d_{ij}Q_iH_d\bar{D}_j \nonumber\\
&&+ \lambda_{ijk} L_iL_j\bar{E}_k + \lambda'_{ijk} L_iQ_j\bar{D}_k + \lambda''_{ijk}\bar{U}_i\bar{D}_j\bar{D}_k.
\end{eqnarray}
The parameter $\mu$, potentially complex, is a brand new supersymmetric parameter appearing in the superpotential, with no corresponding parameter existing in the Standard Model Lagrangian. However, the Yukawa couplings $y_{ij}$ are identical to the Standard Model Yukawa couplings since they will be required to give mass to the Standard Model fermions after electroweak symmetry breaking when the Higgs fields get a vev, see Section~\ref{sec:rewsb}. The fate of the other parameters will be discussed in the next section.



%%%%%%%%%
\section{R-parity}
%%%%%%%%%
The superpotential terms $LH_u$, $LLE$ and $LQ\bar{D}$ that we have written down in Eq.~(\ref{eq:supolag}) all violate lepton number conservation, and $\bar{U}\bar{D}\bar{D}$ violates baryon number conservation. Such terms do not exist in the Standard Model, although there is no symmetry forbidding their existence there. Instead a seemingly accidental combination of what fields exist and the gauge symmetries that limit their interactions means that there are no such tree-level interactions. We call this an {\bf accidental symmetry} of the Standard Model.  

Allowing such terms in the MSSM would lead to, among other phenomenological problems, processes like rapid proton decay, for example through $p\to e^+ \pi^0$ as shown in Fig.~\ref{fig:proton}, which breaks both baryon number and lepton number. These are not observed in nature. We can estimate the resulting proton life-time by noting that the exchange of a scalar particle (in this example a strange squark $\tilde s$) creates an effective dimension-6 four-fermion interaction Lagrangian term $\lambda\bar u \bar deu$ with coupling 
\begin{equation}
\lambda = \frac{\lambda'_{112}\lambda''_{112}}{m_{\tilde{s}}^2},
\end{equation}
where the sparticle mass $m_{\tilde s}$ comes from the scalar propagator in the diagram. The resulting matrix element for the total proton decay process must then be proportional to $|\lambda|^2$, which has mass dimension $M^{-4}$. Since decay width has mass dimension $M$, the phase space part of the calculation must provide something of mass dimension $M^5$.
The only mass scale involved in the problem is the proton mass $m_p$, thus we approximate the phase space integration part of the proton decay width by $m_p^5$. We then have
\begin{equation}
\Gamma_{p\to e^+ \pi^0} \sim |\lambda|^2 m_p^5 = |\lambda'_{112}\lambda''_{112}|^2\frac{m_p^5}{m_{\tilde{s}}^4}.
\end{equation}

%%%
\begin{figure}[h]
\begin{center}
\includegraphics{figures/protondecay.eps}
\caption{Possible Feynman diagram for proton decay with R-parity violating couplings $\lambda_{112}^{\prime\prime}$ and $\lambda_{112}'$.\label{fig:proton}}
\end{center}
\end{figure}
%%%

%%%%
%\begin{figure}[h]
%\centering
%\begin{tikzpicture}
%\begin{feynman}
%%
%\diagram [horizontal=b to c] {
%a [particle=$d$] -- [fermion] b -- [scalar] c -- [anti fermion] d [particle=$e^+$],
%e [particle=$u$] -- [fermion] b ,
%c -- [fermion] f [particle=$\bar u$],
%d --  [opacity=0] f,
%a --  [opacity=0] e,
%};
%\diagram [yshift=1cm, horizontal=a to b] {
%a [particle=$u$] -- [fermion] b [particle=$u$],
%};
%%\path (b)--++(90:0.8) coordinate (A);
%%\draw [dashed] (A) circle(0.8);
%%
%\end{feynman}
%\end{tikzpicture}
%\caption{The tree level coupling between three gauge bosons $B$ (left), and the one-loop fermion contribution to the same process (right).}  
%\label{fig:proton}
%\end{figure}
%%%%

The current measured lower limit on the lifetime from watching a lot of protons not decay is $\tau_{p\to e^+\pi^0}> 1.6\cdot 10^{34}~{\rm y}$~\cite{Super-Kamiokande:2016exg}, or, converting to seconds, $ \tau_{p\to e^+\pi^0}> \pi\cdot 10^7~{\rm s/y} \times1.6\cdot 10^{34}~{\rm y} = 5.0\cdot 10^{41}$\,s, which gives a limit on the width
\[\Gamma_{p\to e^+ \pi^0}=\frac{\hbar}{\tau}< \frac{6.582\cdot 10^{-25}\,\text{GeV s}}{5.0\cdot 10^{41}\,\text{s}} \simeq 1.3 \cdot 10^{-66}\,\text{GeV}, \]
so that we have the following very strict limit on the combination of the two couplings
\begin{equation}
|\lambda'_{112}\lambda''_{112}|< 1.3 \cdot 10^{-27}\left(\frac{m_{\tilde{s}}}{1~{\rm TeV}}\right)^2.
\end{equation}
Such strict limits can be found on most of the lepton and baryon number couplings using the measured properties of the Standard Model particles, with some exceptions for coupling involving the second and third generation fermions.

To avoid all such lepton and baryon number couplings Fayet (1975) \cite{Fayet:1975ki} introduced the conservation of R-partity.
\df{{\bf R-parity} is a multiplicatively conserved quantum number given by
\[R = (-1)^{2s + 3B + L}\]
where $s$ is a particle's spin, $B$ its baryon number and $L$ its lepton number.}
For all Standard Model particles, including all the scalar Higgs bosons, this gives $R=1$, while the superpartners all have $R=-1$. One usually {\it defines} the MSSM as conserving R-parity. For the MSSM this excludes the terms $L H_u$, $LL\bar{E}$, $LQ\bar{D}$ and $\bar{U}\bar{D} \bar{D}$ from the superpotential,\footnote{All the superpotential Yukawa terms lead to component field terms of the form $A_i\psi_j\psi_k$. If the scalar $A_i$ here is {\it not} a Higgs boson, then it is a superpartner and if none of the fermions come from a Higgs superfield so that they are also a superpartner the term breaks R-parity conservation. This means that every Yukawa term needs one, and only one, Higgs superfield to conserve R-parity.  The superpotential mass terms have component field terms of the form $\psi_i\psi_j$. If one of the fermions here comes from a Higgs superfield, then it is a superpartner, and if the other does not, the term breaks R-parity.}
leaving us with the R-partiy conserving superpotential
\begin{eqnarray}
W &=& \mu H_u H_d + y^e_{ij}L_iH_d E_j + y^u_{ij}Q_iH_u\bar{U}_j +  y^d_{ij}Q_iH_d\bar{D}_j.
\label{eq:RPCsuperpot}
\end{eqnarray}

The consequence of this somewhat {\it ad hoc} definition is that in all interactions the total number of incoming and outgoing supersymmetric particles must be an even number $2n$, so that the total R-number is $(-1)^{2n}=1$. This leads to the following very important phenomenological consequences:
\begin{enumerate}
\item All sparticles must be produced in pairs (ignoring the very low probability of producing four or more).
\item Sparticles must annihilate in pairs.
\item The {\bf lightest supersymmetric particle} (LSP) is absolutely stable, and every other sparticle must decay down to the LSP (possibly in multiple steps).
\end{enumerate}

%TODO:  Example of  broken $U_{B-L}$ giving $\mathbb Z_2$ symmetry (Melkild), also connection to anomalies (extra neutrinos=


%%%%%%%%%%%%%%%
\section{Supersymmetry breaking terms in the MSSM}
\label{sec:MSSM_soft_terms}
%%%%%%%%%%%%%%%
We can directly apply our previous arguments on gauge invariance, that we used when discussing the superpotential, on the general soft-breaking terms in Eq.~(\ref{eq:gensoftterms}) in order to determine which supersymmetry breaking terms are allowed in the MSSM, keeping also in mind the requirement of R-party conservation.

Mass terms of the form
\[ -\frac{1}{4T(R)q^2} M\theta\theta\bar{\theta}\bar{\theta}\tr{W^AW_A}+\text{c.c.}, \]
are allowed because they have the same gauge structure as the supersymmetric field strength terms. In component fields only the fermions in the vector superfields survive, and are for the MSSM:
\[\mathcal L_\text{soft} =  -\frac{1}{2}M_1\tilde{B}^0\tilde{B}^0 - \frac{1}{2}M_2\tilde{W}^i \tilde{W}^i- \frac{1}{2}M_3 \tilde{g}^{a}\tilde{g}^a + \text{c.c.},\]
where $M_i\in\mathbb C$. This gives six new parameters. 

Yukawa terms
\[-\frac{1}{6} a_{ijk}\theta\theta\bar{\theta}\bar{\theta} \Phi_i\Phi_j\Phi_k+\text{h.c.},\]
are allowed when a corresponding term exist in the superpotential  -- meaning they are gauge invariant. In component fields only the scalar parts of the superfields survive, and the allowed terms are
\[ \mathcal L_\text{soft} =  -a^e_{ij}\tilde{L}_iH_d\tilde{l}^*_{jR} - a_{ij}^u \tilde{Q}_i H_u \tilde{u}^*_{jR} - a_{ij}^d \tilde{Q}_i H_d \tilde{d}^*_{jR} + \text{c.c.},\]
where we remind you that the $H$ here refers to scalar parts of the Higgs superfield doublets,
\[ H_d=\left( \begin{matrix} H_u^+ \\ H_u^0 \end{matrix}\right) \quad  \text{and} \quad H_d=\left(\begin{matrix} H_d^0 \\ H_d^- \end{matrix}\right),\]
and
\[ \tilde L_i =\left(\begin{matrix} \tilde\nu_{iL} \\ \tilde l_{iL} \end{matrix}\right) \quad\text{and}\quad \tilde Q_i =\left(\begin{matrix} \tilde u_{iL} \\ \tilde d_{iL}\end{matrix}\right),\] 
in the normal $SU(2)_L$ invariant construction.
We can see that all of these terms are clearly R-parity conserving, since they consist of two sparticles and one (Higgs) particle.
The couplings $a_{ij}$ are all potentially complex valued, so this gives us 54 new parameters. 

The mass terms
\[-\frac{1}{2}b_{ij}\theta\theta\bar{\theta}\bar{\theta}\Phi_i \Phi_j+\text{h.c.},\]
are again only allowed for corresponding terms in the superpotential, {\it i.e.}\ 
\[ \mathcal L_\text{soft} = -bH_uH_d + \text{c.c.},\]
where $b$ is potentially complex valued, which gives us 2 new parameters.\footnote{The coupling $b$ is sometimes written $B\mu$ where $B$ is a factor that indicates how different the coupling is from the corresponding coupling in the superpotential.} Tadpole terms are not allowed, as there are no tadpoles in the superpotential. 

Mass terms
\[-m_{ij}^2\theta\theta\bar{\theta}\bar{\theta}\Phi^\dagger_i \Phi_j,\]
are allowed because they have the same gauge structure as the supersymmetric kinetic terms. In component fields again only the scalar fields survive, and in the MSSM they are:
\begin{eqnarray}
\mathcal L_\text{soft} &=& -(m^L_{ij})^2\tilde{L}_i^\dagger\tilde{L}_j -(m^e_{ij})^2\tilde{l}_{iR}^*\tilde{l}_{jR} - (m_{ij}^Q)^2\tilde{Q}_i^\dagger \tilde{Q}_j - (m^u_{ij})^2\tilde{u}^*_{iR}\tilde{u}_{jR} - (m_{ij}^d)^2\tilde{d}^*_{iR}\tilde{d}_{jR}\nonumber\\
&& - m_{H_u}^2H_u^\dagger H_u - m_{H_d}^2 H_d^\dagger H_d,
\label{eq:MSSM_soft_kin}
\end{eqnarray}
where the $m_{ij}^2$ are potentially complex valued, however, also hermitian. This gives rise to 47 new parameters. Despite technically being allowed the MSSM ignores the ``maybe-soft" terms in Eq.~(\ref{eq:maybesoft}).

In total, after using our freedom to choose our basis for the fields wisely in order to remove what freedom we can, the MSSM has 105 new parameters compared to the Standard Model, 104 of these are soft-breaking terms and $\mu$ is the only new parameter in the superpotential.



%%%%%%%%%%%%%%%%%%%%%%
\section{Renormalisation group equations}
\label{sec:RGE}
%%%%%%%%%%%%%%%%%%%%%%
Renormalisation, the removal of infinities from field theory predictions, introduces a fixed scale $\mu$ at which the fields and the parameters of the Lagrangian, the couplings, are defined. For example, the charge of the electron is not simply the {\bf bare charge} $e_0$ given in the original Lagrangian, but a charge at a given energy scale $\mu$, $e(\mu)$, which is the scale at which the theory wants to describe the electron, and which we can measure in an experiment at that scale. Describing the scattering an electron at very high energy will require a different value of $e(\mu)$ than at a low energy. This scale dependence in the coupling is an experimentally well verified fact.\footnote{It is also impossible to avoid if we accept that the electron is a point particle. Since the potential has the form $V(r)\propto e/r$ an infinite energy would appear unless we were somehow to modify the charge at high energies, or equivalently, short distances.} The relationship between a bare field or coupling, and the (dressed) renormalised field or coupling can be found from the so-called {\bf renormalisation constant} $Z$ that renormalises the parameter. For example for a field $\phi$, $\phi=Z_\phi \phi_0$, and for coupling $g$, $g=Z_gg_0$.

However, since $\mu$ is not an observable {\it per se} but in principle a choice of how to write down the theory to compare to an experiment (at which energy scale to write down the Lagrangian), the renormalised effective action $S$ for a physical process should be invariant under a change of $\mu$, which is expressed as:\footnote{It is more common in the literature to find this expressed in terms of the Green's function $G^{(n)}$ for a given $n$-point correlation, {\it i.e.}\ a process with $n$-field insertions. In this form the equation is known as the {\bf Callan--Symanzik equation}.}
\begin{equation}
\mu \frac{d}{d\mu}S(\Phi, \lambda, \mu) = 0,
\end{equation}
where $\lambda$ is a generic name for the couplings of the theory and $\Phi$ represents the (super)fields that have been renormalised.\footnote{In Sec.~\ref{sec:vacuum_energy} we mentioned how the non-renormalisation theorem implies that we do not need to renormalise the coupling constants of the superpotential separately. In the MSSM this is the $\mu$ coupling (not to be confused with the energy scale here also called $\mu$) and the Yukawa couplings. This will now have the consequence that their renormalisation can be expressed in terms of the renormalisation of the fields.} This equation can be re-written in terms of partial derivatives
\begin{equation}
\left(\mu \frac{\partial}{\partial \mu} + \beta_\lambda\frac{\partial}{\partial\lambda}+n_\Phi\gamma_\Phi\right)S(\Phi, \lambda, \mu)=0,
\end{equation}
which is the {\bf renormalisation group equation} (RGE). Here $\beta_\lambda$ is the {\bf $\boldsymbol \beta$-function}:
\begin{equation}
\beta_\lambda\equiv\mu\frac{\partial \lambda}{\partial \mu}.
\end{equation}
which describes the behaviour of a Lagrangian parameter $\lambda$ as a function of the energy scale $\mu$ away from the value where it was defined, often denoted $\mu_0$. The {\bf anomalous dimension} $\boldsymbol\gamma$ describes the scaling of the fields and the factor $n$ gives the number of fields (of a given kind) in the effective action. The RGE balances the different contributions to sum to zero.

The reason for keeping around the factor of $\mu$ in the definition of the $\beta$-function is that it typically changes very slowly over large differences in energy scale, so it is practical to change variable to $t = \ln\frac{\mu}{\mu_0}$, so that $\mu\frac{\partial}{\partial\mu}=\frac{\partial}{\partial t}$ and $\beta_\lambda = \frac{\partial \lambda}{\partial t}$.
If the $\beta$-functions of a quantum field theory are zero at some value of the couplings, then the value of the theory is said to be {\bf scale-invariant}. 
%Almost all scale-invariant QFTs are also conformally invariant. The study of such theories is conformal field theory.
%The coupling parameters of a quantum field theory can run even if the corresponding classical field theory is scale-invariant. In this case, the non-zero beta function tells us that the classical scale invariance is anomalous. 

As an example of finding a $\beta$-function, take the relationship between a bare {\bf gauge coupling constant} $g_0$ and the renormalised coupling $g$. This is given by (in $d= 4-\epsilon$ dimensions):\footnote{The factor $\mu^{-\epsilon/2}$ is there to ensure that the scale of $g_0$ is correct.}
\[g_0 = Z_gg\mu^{-\epsilon/2}.\]
Then, differentiating both sides with respect to $\mu$,
\begin{eqnarray*}
0 &=& \frac{\partial Z_g}{\partial \mu} g\mu^{-\epsilon/2} + Z_g\frac{\partial g}{\partial \mu}\mu^{-\epsilon/2}-\frac{\epsilon}{2}Z_gg\mu^{-\epsilon/2-1}\\
\mu \frac{\partial g}{\partial \mu} &=&\frac{\epsilon}{2}g - \frac{g\mu}{Z_g}\frac{\partial Z_g}{\partial \mu}\\
\mu \frac{\partial g}{\partial \mu} &=&\frac{\epsilon}{2}g - g\mu \frac{\partial }{\partial \mu}\ln Z_g,
\end{eqnarray*}
and taking the limit $\epsilon \to 0$:
\[\beta_g \equiv \mu\frac{\partial g}{\partial\mu} = -g\gamma_g,\]
where we have defined the {\bf anomalous dimension} of $g$
\begin{equation}
\gamma_g =\mu \frac{\partial}{\partial\mu}\ln Z_g.
\end{equation}


The renormalisation constant $Z_g$ can now be calculated to the required loop-order to find the $\beta$-function to that order, and in turn the running of the coupling constant with $\mu$. By evaluating to one-loop order we can find that for our particular example of a gauge coupling constant for a generic supersymmetric model
\begin{equation}
\gamma_g\left|_{\rm 1-loop}\right. = -\frac{1}{16\pi^2}\,g^2\left(\sum_R T(R) - 3C(A)\right), 
\label{eq:ccbeta}
\end{equation}
where the sum of Dynkin indices $T(R)$ is over all superfields that transform under a representation $R$ of the gauge group in question, and $C(A)$ is the {\bf quadratic Casimir invariant} of the adjoint representation $A$ of the vector field under the gauge group
\[ C(A)\delta_{ij}=(T^aT^a)_{ij}. \]
For the adjoint representation of $U(1)$ this is 0, and for $SU(N)$ this is $N$. The running of the coupling constants is particularly important since it will later lead us to the concept of gauge coupling unification. 

%Notice both that the running of the couplings with scale $\mu$ is very slow because the $\beta$-function is a logarithmic function of $\mu$ and that the anomolous dimension may be negative for some gauge groups.

%For the parameters in the superpotential the non-renormalisation theorem discussed in Sec.~\ref{sec:vacuum_energy} implies that these do not need separate renormalisation. This means that the anomalous dimensions for these are the same as the anomalous dimensions for the superfields. The $\beta$-functions are
%\begin{eqnarray}
%\beta_g
%\end{eqnarray}

As a second relevant example, for the soft-breaking parameters $M_i$ in (\ref{eq:gensoftterms}) we have the one-loop  $\beta$-functions
\begin{equation}
\beta_{M_i}\equiv \frac{d}{dt}M_i = \frac{1}{16\pi^2}g_i^2M_i\left(2\sum_R T(R)-6C(A)\right).
\label{eq:Mi_beta}
\end{equation}



%%%%%%%%%%%%%%%%%%
\section{Gauge coupling unification}
\label{sec:GUT}
%%%%%%%%%%%%%%%%%%
The one-loop $\beta$-functions for gauge couplings in a generic supersymmetric model were given in Eq.~(\ref{eq:ccbeta}). With the MSSM field content and the gauge couplings discussed in this chapter:\footnote{The normalisation choice for $g_1$ may seem a bit strange, however, this is the correct numerical factor when for example breaking a unified group such as SU(5) or SO(10) down to the Standard Model gauge group. This factor might be different with a different unified group.} 
\[g_1 = \sqrt{\frac{5}{3}}g', \indent g_2 = g, \indent g_3=g_s,\]
we arrive at
\begin{equation}
\beta_{g_i}\left|_{\rm 1-loop}\right. = \frac{1}{16\pi^2}\,b_ig_i^3,
\end{equation}
with in the MSSM
\[b_i^\text{MSSM} = \left(\frac{33}{5}, 1, -3\right).\]
For comparison, the same result in the Standard Model is
\[b_i^\text{SM} = \left(\frac{41}{10}, -\frac{19}{6}, -7\right).\]

The values of $b_i$ for the MSSM are found from the Casimir invariant and the Dynkin index of the gauge group representations
\[C(A)_{SU(3)} = 3, \quad C(A)_{SU(2)} = 2, \quad C(A)_{U(1)} = 0,\]
and
\[T(R)_{SU(3)} = \frac{1}{2}, \indent T(R)_{SU(2)} = \frac{1}{2}, \indent T(R)_{U(1)} = \frac{3}{5}Y^2,\]
where for example $b_3 = \frac{1}{2}\cdot 12 - 3\cdot 3 = -3$ in the MSSM, because, after careful counting, we have twelve quark/squark scalar superfields transforming under $SU(3)_C$.

At one-loop order we can do a neat rewrite using $\alpha_i\equiv\frac{g_i^2}{4\pi}$. Since \[\frac{d}{dt}\alpha^{-1}_i = -2\frac{4\pi}{g_i^3}\frac{d}{dt}g_i,\] we have:
\[\beta_{\alpha_i^{-1}} \equiv \frac{d}{dt}\alpha^{-1}_i = -\frac{8\pi}{g_i^3}\frac{1}{16\pi^2}g_i^3b_i = -\frac{b_i}{2\pi}.\]
Thus $\alpha^{-1}$ runs linearly with $t$ at one loop. 

By running the couplings $\alpha_i^{-1}$ from their values measured at the electroweak scale to high energies it is observed that in the MSSM the coupling constants intersect at a single point, which they do not naturally do in the Standard Model. See Fig.~\ref{fig:unification}, taken from Martin~\cite{Martin:1997ns}. The common assumption made is then that a unified gauge group, {\it e.g.}\ $SU(5)$ or $SO(10)$, is broken at that scale, called the {\bf grand unification theory} scale or GUT-scale, down to the Standard Model gauge group. This scale is $\mu_\text{GUT} \approx 2\cdot 10^{16}$\,GeV, about two orders of magnitude below the Planck scale. 

\begin{figure}[h]
\centering
\includegraphics[width=0.8\textwidth]{figures/unification.eps} 
\caption{The RGE evolution of the inverse gauge couplings $\alpha^{-1}_i(Q)$ in the Standard Model (dashed lines) and the MSSM (solid lines). In the MSSM case, the sparticle mass thresholds are varied between 250 GeV and 1 TeV and $\alpha_3(m_Z)$ between 0.113 and 0.123 to create the bands shown by the red and blue lines. Two-loop effects are included.}
\label{fig:unification}
\end{figure}

Something funny happens to the gaugino soft-mass parameters $M_i$ if we look at their running. From (\ref{eq:Mi_beta}) the one-loop $\beta$-functions for the $M_i$ in the MSSM are
\begin{equation}
\beta_{M_i}|_{\rm 1-loop} \equiv \frac{d}{dt}M_i = \frac{1}{8\pi^2}g_i^2M_ib_i.
\end{equation}
As a consequence all three ratios $M_i/g_i^2$ are scale independent at one loop. To see this let $R=M_i/g_i^2$, then
\begin{equation}
\beta_R \equiv \frac{dR}{dt}= \frac{\frac{dM_i }{dt}g_i^2 - M_i \frac{d}{dt} g_i^2}{g_i^4} = \frac{\frac{1}{8\pi^2}g_i^2M_ib_i\cdot g_i^2 - M_i\cdot 2g_i\cdot\frac{1}{16\pi} g_i^3 b_i}{g_i^4} = 0.
\end{equation}
In other words, $R$ does not change with scale $t$.

If we now use that the coupling constants unify at the GUT scale to the coupling $g_u$, and assume that the soft-masses are the same at that scale $m_{1/2} = M_1(\mu_\text{GUT}) = M_2(\mu_\text{GUT}) = M_3(\mu_\text{GUT})$,\footnote{Again, not unreasonable if the spontaneous symmetry breaking mechanism acts uniformly for all the gauginos.} it follows that
\begin{equation}
\frac{M_1}{g_1^2} = \frac{M_2}{g_2^2} = \frac{M_3}{g_3^2} = \frac{m_{1/2}}{g_u^2},
\end{equation}
at all scales!\footnote{At one-loop level.} This is a very powerful and predictive assumption. Because of the relationship between the electroweak couplings and the electric charge, $e=g'\cos\theta_W=g\sin\theta_W$, it leads to the following relation
\begin{equation}
M_3 = \frac{\alpha_s}{\alpha}\sin^2\theta_WM_2 = \frac{3}{5}\frac{\alpha_s}{\alpha}\cos^2\theta_W M_1,
\end{equation}
which, inserting values for the fine structure constant, the strong coupling, and the Weinberg angle, numerically predicts
\[M_3:M_2:M_1 \simeq 6:2:1\]
at the electroweak scale. We will return to the implications of this when discussing the gauginos in Sec.~\ref{sec:electroweakinos}.

In Fig.~\ref{fig:MSSMrun}, again taken from Martin~\cite{Martin:1997ns}, we show the running of all the types of soft parameters in the MSSM. We assume unified soft-mass parameters $m_{1/2}$ for the gauginos and $m_0$ for the Higgs and sfermions at the GUT scale. Shown is the gaugino mass parameters $M_i$ (solid black), the Higgs mass parameters $m_{H_{d/u}}^2$ (dot-dashed green), the third generation sfermion soft terms $m_{d_3}$, $m_{Q_3}$, $m_{u_3}$, $m_{L_3}$ and $m_{e_3}$ (dashed red and blue, listed from top to bottom), and the corresponding first and second generation terms (solid red and blue lines).
%%%%%%%%% 
\begin{figure}[h]
\centering
\includegraphics[width=0.8\textwidth]{figures/MSSMrun.eps} 
\caption{The RGE evolution of soft-mass parameters in the MSSM with typical minimal supergravity-inspired boundary conditions imposed at $2\cdot10^{16}$\,GeV. The parameter values
used for this illustration were $m_0 = 200$\,GeV, $m_{1/2 }= -A_0 =600$\,GeV, $\tan\beta = 10$, and ${\rm sgn}(\mu)=+$.}
\label{fig:MSSMrun}
\end{figure}
%%%%%%%%%%



%%%%%%%%%%%%%
\section{Radiative Electroweak Symmetry Breaking}
\label{sec:rewsb}
%%%%%%%%%%%%%
In the Standard Model the gauge symmetries prevent mass terms for both vector bosons and fermions. The $W$ and $Z$ bosons, and all the fermions are given mass by spontaneously electroweak symmetry breaking (EWSB), which is induced by the shape of the scalar potential for the Higgs field $H$, which is an $SU(2)_L$ doublet of scalar fields:
\begin{equation}
V(H) = \mu^2 |H|^2+ \lambda |H|^4,
\label{eq:SMscalarpot}
\end{equation}
with $|H|^2=H^\dagger H$. 

Here, the requirement for successful EWSB is that $\lambda > 0$ and $\mu^2 < 0$.\footnote{This is called the {\bf Mexican hat} or {\bf wine bottle potential}, depending on preferences.} The first of these is a consistency requirement that ensures that the potential is {\bf bounded from below}, {\it i.e.}\ that in the limit of large field values the potential does not turn to negative infinity. The second ensures that the minimum of the potential, the vacuum, is not given by zero field values, {\it i.e.}\ that the Higgs field has a {\bf vacuum expectation value} (vev).

In supersymmetry we found the following general scalar potential in Eq.~(\ref{eq:scalarpotglobal}) for unbroken supersymmetry,
\begin{equation}
V(A, A^*) = \sum_i \left|\frac{\partial W}{\partial A_i}\right|^2 + \frac{1}{2}\sum_a g^2(A^*T^aA)^2 >0,
\end{equation}
where the first part is due to the elimination of the auxiliary $F$-fields in the scalar superfields, while the second part is due to the elimination of the auxiliary $D$-fields in the vector superfields. In addition we have to add all terms containing only relevant scalar fields from the soft breaking terms in Eq.~\ref{eq:soft_terms_component_fields}.

For the scalar Higgs component fields in the MSSM this gives the potential
\begin{eqnarray}\label{eq:scaH}
V(H_u,H_d) &=& |\mu|^2 (|H_u^0|^2 + |H_u^+|^2 + |H_d^0|^2 + |H_d^-|^2)  \text{\indent \indent\indent \indent\indent\indent\indent}\text{(from $F$-terms)} \nonumber\\
&&+\frac{1}{8} (g^2 + g'{}^2)(|H^0_u|^2 + |H_u^+|^2 - |H^0_d|^2 - |H_d^-|^2)^2  \text{\indent\indent\indent\indent(from $D$-terms)} \nonumber\\
&&+ \frac{1}{2}g^2|H^+_uH_d^0{}^* + H^0_u H^-_d{}^*|^2 \nonumber\\
&&+ m_{H_u}^2(|H_u^0|^2 + |H_u^+|^2) + m_{H_d}^2(|H_d^0|^2 + |H_d^-|^2) \text{\indent}\text{(from soft breaking terms)}\nonumber\\
&& + [b(H_u^+H_d^- - H_u^0H_d^0) + \text{c.c.}]
\label{eq:MSSMscalarpot}
\end{eqnarray}
This potential has 8 d.o.f.\ from 4 complex scalar fields $H_u^+$, $H_u^0$, $H_d^0$ and $H_d^-$. At the same time it has 6 parameters: the two Standard Model gauge couplings $g$ and $g'$, the magnitude of the supersymmetric parameter $\mu$, and the three soft-breaking parameters $b$, $m_{H_u}^2$ and $m_{H_d}^2$. Notice how if $b=m_{H_u}^2=m_{H_d}^2=0$, meaning no supersymmetry breaking terms, the Higgs potential has a global minimum at $V=0$ for zero values of all the Higgs fields. In this case there is no EWSB, so EWSB is intimately connected to supersymmetry breaking in the MSSM.

We now want to do as in the Standard Model and break $SU(2)_L\times U(1)_Y \to U(1)_{\rm em}$ in order to give masses to gauge bosons and SM fermions.\footnote{You may ask why we can not use the soft-terms from the spontaneous breaking of {\it supersymmetry} to do this. However, the soft-terms are unable to effectively provide masses to vector bosons and fermions because they deal (mostly) with scalar fields.} To do this we need to show that, and under which conditions, Eq.~(\ref{eq:MSSMscalarpot}) has: i) a minimum for finite, {\it i.e.}\ non-zero, field values, ii) that this minimum has a remaining $U(1)_{\rm em}$ symmetry and iii) that the potential is bounded from below. We will here restrict our analysis to tree level, ignoring loop effects on the potential.

We start by using our $SU(2)_L$ gauge freedom, picking a gauge so that we rotate away any field value for $H_u^+$ at the minimum of the potential. So without loss of generality we can use that $H_u^+ = 0$ at the minimum in what follows. At the minimum we must also have $\partial V/\partial H_u^+= 0$ since it is a minimum, and by explicit differentiation of the potential one can show that $H_u^+ = 0$ then also leads to $H^-_d = 0$. This is good and proper since it guarantees our item ii), that $U(1)_{\rm em}$ is a symmetry for the minimum of the potential, since the charged fields then have no vevs. 

We are now left with the following potential only in terms of the uncharged Higgs fields $H_u^0$ and $H_d^0$ (after the $SU(2)_L$ gauge choice and at the minimum):
\begin{eqnarray}
V(H_u^0,H_d^0) &=& (|\mu|^2 + m_{H_u}^2)|H^0_u|^2 + (|\mu|^2 + m_{H_d}^2)|H^0_d|^2 \nonumber\\
&&+ \frac{1}{8}(g^2 + g'{}^2)(|H_u^0|^2 - |H_d^0|^2)^2 - (bH_u^0H_d^0 + \text{c.c.})
\label{eq:higgspot_gauged}
\end{eqnarray}
We can now absorb a complex phase in $H^0_u$ or $H_d^0$, in order to take $b$ to be real and positive. This does not affect other terms because they are protected by absolute values. The minimum must also have the total phase of $H_u^0H_d^0$ real and positive, to get an as large as possible negative contribution from the $b$ term, which is the only term that can be negative. Thus the vevs $v_u \equiv\langle H_u^0\rangle$ and $v_d \equiv \langle H_d^0\rangle$ must have opposite phases. By the remaining $U(1)_Y$ gauge symmetry of the potential, which is effectively a phase rotation, and the fact that $H^0_u$ and $H_d^0$ have opposite hypercharge, we can transform $v_u$ and $v_d$ so that they are real and have the same sign. For the potential to have a negative mass term, and thus fulfill point i) above, we must then have
\begin{equation}\label{eq:higgsbound2}
b^2 > (|\mu|^2 + m_{H_u}^2)(|\mu|^2 + m_{H_d}^2).
\end{equation}

Since we have broken supersymmetry we must also check that the potential is actually bounded from below, our point iii), which was guaranteed for a supersymmetric vacuum. For large $|H_u^0|$ or $|H_d^0|$ the quartic gauge terms in (\ref{eq:higgspot_gauged}) blows up to save the potential, except for $|H^0_u| = |H^0_d|$, the so-called $D$-flat directions. This means that we must also require
\begin{equation}\label{eq:higgsbound1}
2b < 2|\mu|^2 + m_{H_u}^2 + m_{H_d}^2.
\end{equation}

To summarise what we have learnt so far: at the minimum of the Higgs potential we know that there exists a gauge choice so that the expectation values of the charged Higgs component fields are zero, $\langle H_u^+\rangle =0$ and $\langle H_d^-\rangle =0$, and we fulfil the condition for the existence of an extremal point in the neutral Higgs component fields
\begin{equation}
\frac{\partial V}{\partial H_u^0}=\frac{\partial V}{\partial H_d^0}= 0.
\label{eq:EWSB_condition}
\end{equation}
In addition, for the minimum to have non-zero field values that break EWSB, and for the potential to be bounded from below, the parameters of the potential must simultaneously fulfil the inequalities
\begin{eqnarray}
b^2 &>& (|\mu|^2 + m_{H_u}^2)(|\mu|^2 + m_{H_d}^2), \nonumber\\
2b &<& 2|\mu|^2 + m_{H_u}^2 + m_{H_d}^2. \nonumber
\end{eqnarray}
The resulting non-zero expectation values at the minimum for the neutral Higgs component fields are denoted $v_u$ and $v_d$.

To satisfy (\ref{eq:higgsbound2}) and (\ref{eq:higgsbound1}),  a negative value for $m_{H_u}^2$ (or $m_{H_d}^2$) can help, in particular if $|\mu|^2+m_{H_u}^2<0$, as that automatically fulfils (\ref{eq:higgsbound2}). Such a negative value is indeed perfectly allowed as a parameter in the Lagrangian. No negative particle masses will result. 

In fact, if we assume that $m_{H_d} = m_{H_u}$ at some scale  $\mu$, for example the GUT scale, then \eqref{eq:higgsbound2} and \eqref{eq:higgsbound1} cannot be simultaneously be satisfied {\it at that scale}. However, to 1-loop the RGE running of these mass parameters is:
\[16\pi^2 \beta_{m_{H_u}^2} \equiv 16\pi^2 \frac{dm_{H_u}^2}{dt} = 6|y_t|^2(m_{H_u}^2 + m_{Q_3}^2 + m_{u_3}^2) + ...\]
\[16\pi^2 \beta_{m_{H_d}^2} \equiv 16\pi^2 \frac{dm_{H_d}^2}{dt} = 6|y_b|^2(m_{H_d}^2 + m_{Q_3}^2 + m_{d_3}^2) + ...,\]
where $y_t$ and $y_b$ are the top and bottom quark Yukawa couplings, and $m_{Q_3} = m_{33}^Q$, $m_{u_3}  = m_{33}^u$, and $m_{d_3} = m_{33}^d$, in a simplification of our previous notation for the soft-masses in Sec.~\ref{sec:MSSM_soft_terms}. Because $y_t\gg y_b$, $m_{H_u}^2$ runs much faster with scale than $m_{H_d}^2$. If the parameters start out the same at some high scale, say from some universal supersymmetry breaking effect,  as we go down to the electroweak scale  $m_{H_u}^2$ becomes significantly smaller than  $m_{H_d}^2$, and may become negative, fulfilling the EWSB criteria. For an illustration, see Fig.~\ref{fig:MSSMrun}, where the running of  $\mu^2+m_{H_u}^2$ and  $\mu^2+m_{H_u}^2$ is shown.
It is this property of starting our from some universal Higgs parameters at a high scale, which then by RGE effects break electroweak symmetry at lower scales, that is termed {\bf radiative EWSB} (REWSB). Thus, in the MSSM with universal soft terms at a high scale there is an explanation why EWSB happens, it is not put in by hand in the potential as it is in the Standard Model.

%\begin{figure}[h]
%\centering
%\includegraphics[width=0.7\textwidth]{figures/REWSB.eps} 
%\caption{Sketch of the RGE running for the two soft Higgs mass parameters $m_{H_u}^2$ and $m_{H_d}^2$ as a function of the energy scale. The parameter start out identical at $\mu=10^{16}$ GeV. Shown are the actual parameters of the EWSB criteria $\mu^2+m_{H_u}^2$ and  $\mu^2+m_{H_u}^2$.}
%\label{fig:REWSB}
%\end{figure}

Following EWSB, to get the familiar vector boson masses measured by experiment, the vevs need to satisfy the constraint from the electroweak parameters:
\begin{equation}
v_u^2 + v_d^2 \equiv v^2 = \frac{2m_{Z}^2}{g^2 + g'{}^2} \approx (174~\text{GeV})^2.
\label{eq:Zmass_vev_condition}
\end{equation}
Thus we have one free parameter coming from the two Higgs vevs in the MSSM. We can write this as 
\[\tan\beta \equiv \frac{v_u}{v_d},\]
where by convention $0<\beta<\pi/2$, so that $0<\tan\beta<\infty$. 

Using the condition for the existence of an extremal point in (\ref{eq:EWSB_condition}), the two non-SM parameters $b$ and $|\mu|$ can be eliminated as free parameters from the model, however, not the sign of $\mu$. Alternatively, we can choose to eliminate $m_{H_u}^2$ and $m_{H_d}^2$. You can look at this as giving away the freedom of these parameters to the vevs, and then fixing one vev by the electroweak constraint, and using $\tan\beta$ for the other.

Let us make a little remark here on the parameter $\mu$. Given the criteria for REWSB above we have what is called the {\bf $\boldsymbol \mu$ problem}. The soft terms all get their scale from some common mechanism at some common high energy scale, it is assumed, setting the parameters $b$, $m_{H_u}^2$ and $m_{H_d}^2$ in the Higgs potential.  However, $\mu$ is a mass term in the superpotential (the only one in fact) and could {\it a priori} take {\it any} value, even $M_P$. Why is $\mu$ then of the order of the soft terms, which is what allows us to achieve REWSB, when a much larger value would prevent us from fulfilling the criteria in  (\ref{eq:higgsbound2}) and (\ref{eq:higgsbound1})?\footnote{This problem can be solved in extensions of the MSSM such as the Next-to-Minimal Supersymmetric Standard Model (NMSSM).}


%%%%%%%%%%%%%%%%%
\section{Higgs boson properties}
%%%%%%%%%%%%%%%%%
Of the eight d.o.f.\ in the scalar potential for the Higgs component fields three are Goldstone bosons that get eaten by $Z$ and $W^\pm$ to give them masses. The remaining five d.o.f.\ form two neutral scalars $h$, $H$, two charged scalars $H^\pm$ and one neutral pseudo-scalar (CP-odd) $A$.\footnote{In addition to the scalars, we know that the Higgs supermultiplets contain four fermions, $\tilde{H}^0_u$, $\tilde{H}^0_d$, $\tilde{H}^+_u$ and $\tilde{H}^-_d$ (higgsinos). We will see later that these mix with the fermion partners of the gauge bosons (gauginos).} At tree level one can show that these have the masses:
\begin{eqnarray}
m_A^2 &=& \frac{2b}{\sin2\beta} = 2|\mu|^2 + m_{H_u}^2 + m_{H_d}^2,\\
m_{h, H}^2 &=& \frac{1}{2}\left(m_A^2 + m_Z^2 \mp \sqrt{(m_A^2- m_Z^2)^2 + 4m_Z^2m_A^2\sin^22\beta}\right),\\
m_{H^\pm}^2 &=& m_A^2 + m_W^2.
\end{eqnarray}
As a consequence $m_A$ and $\tan\beta$ can be used to parametrise the masses of the Higgs sector (at tree level), and
$H$, $H^\pm$ and $A$ are in principle unbounded in mass since they grow as $b/\sin2\beta$. However, at tree level the lightest Higgs boson is restricted to
\begin{equation}
m_h < m_Z|\cos 2\beta| < 91.2~\text{GeV}.
\end{equation}
In contrast we have the current best measurement of the Higgs boson mass of $m_h=125.10\pm 0.14$\,GeV, combining results from the LHC~\cite{ParticleDataGroup:2020ssz}.

Fortunately, there are large loop-corrections or the MSSM would have been excluded already.\footnote{It is worth pointing out here that the MSSM, despite its many parameters, is a falsifiable theory. For example, had the Higgs boson mass been $\sim15$\,GeV higher, which is perfectly allowed in the Standard Model, the MSSM would have been excluded.} Because of the size of the Yukawa couplings the largest corrections to the mass of the lightest Higgs comes from loops with top quarks and its supersymmetric partners, the scalar top quarks, or stops, $\tilde t_L$ and $\tilde t_R$. See Fig.~\ref{fig:hierarchy} for the relevant Feynman diagrams. In the limit where the mass of the stop quarks are larger than the top, $m_{\tilde{t}_R},m_{\tilde{t}_L}\gg m_t $, and with stop mass eigenstates close to the chiral eigenstates (more on this later), we get the dominant loop correction
\begin{equation}
\Delta m_h^2 = \frac{3}{4\pi^2} \cos^2\alpha\, y_t^2 m_t^2 \ln\left(\frac{m_{\tilde{t}_L}m_{\tilde{t}_R}}{m_t^2}\right),\label{eq:mhchiralstop}
\end{equation}
where $\alpha$ is a mixing angle for $h$ and $H$ with respect to the superfield component fields $H_u^0$ and $H_d^0$, given by
\begin{equation}
\frac{\sin\alpha}{\sin\beta}=-\frac{m_H^2+m_h^2}{m_H^2-m_h^2},
\end{equation}
at tree level.

With this and other corrections the upper bound on the lightest Higgs boson mass is weaker:
\[m_h \leq 135~{\rm GeV},\]
assuming a common sparticle mass scale of around $m_{\rm SUSY} \leq 1$\,TeV. Higher values for the sparticle masses give large fine-tuning and weaken the bound very little because of the logarithm in Eq.~(\ref{eq:mhchiralstop}). The bound can be further weakened by adding extra field content to the MSSM, {\it e.g.}\ as in the NMSSM, but there is an upper perturbative limit of $m_h \approx 150$\,GeV.

It is very interesting to discuss what the Higgs discovery actually implies for low-energy supersymmetry. As can be seen from the above numbers it requires rather large squark masses even in the favourable scenario with $\tan\beta\gg 1$ where the tree level mass is $m_h\sim 90$ GeV. A naive estimate from Eq.~(\ref{eq:mhchiralstop}) gives $m_{\tilde t}> 1$\, TeV. However, this does not take into account possible negative contributions to the Higgs mass from heavy gauginos (fermions in the vector superfields), and possible increases in the stop contribution due to tuning of the mixing of the chiral eigenstates $\tilde t_L$ and $\tilde t_R$, in the mass eigenstates $\tilde t_1$ and $\tilde t_2$.

Since the lightest stop quark is expected to be the lightest squark in scenarios with universal soft masses at some high scale -- the reasoning here is the large downward RGE running of $m_{33}^Q$ from a common squark soft mass at some high scale due to the large top Yukawa coupling -- the expected sparticle spectrum lies mostly above 1 TeV, with the possible exception of gauginos/higgsinos. This points to so-called {\bf Split-SUSY} scenarios with heavy scalars and light gauginos, and a relatively large degree of fine-tuning. If one can live with this little hierarchy problem, it will explain why no signs of supersymemtry have been seen yet at the LHC. 

If you are willing to accept fine-tuning of the stop mixing instead, or come up with a good reason for why the mixing should be just-so to give a maximal Higgs mass, you can keep fairly light stop quarks. With the addition of light higgsinos and a light gluino the model is then technically natural, these scenarios are called {\bf Natural SUSY} and could be within the current or near future reach of the LHC. The problem with these models, as we shall see, is that the higgsinos are degenerate, and thus difficult to detect.

%In Split-SUSY scenarios with a neutralino dark matter candidate (see below) the lightest neutralino typically has a significant higgsino component. This means that its should be relatively accessible in direct detection experiments due to its large coupling to normal matter, and in the indirect search for neutrinos from captured dark matter annihilation in the Sun. Both types of experiments may very soon see first indications of a signal if this scenario is indeed realised in nature.

To do calculations with the Higgs bosons in the MSSM we need the Feynman rules that result from the relevant Lagrangian terms. Since these have been listed elsewhere we will not repeat them here, but recommend in particular the PhD-thesis of Peter Richardson~\cite{Richardson:2000nt}, where they can be found in Appendix A.6, including all interactions with fermions and sfermions. These can also be found, together with all gauge and self-interactions, in the classic paper by Gunion and Haber~\cite{Gunion:1984yn}. Note that in this paper a complex Higgs singlet appears in some interactions because they perform their calculations in the NMSSM, but this can safely be ignored and all other results carry over into the MSSM.


%%%%%%%%%%%%%%%
\section{The gluino}
%%%%%%%%%%%%%%%
The fermion partner of the Standard Model gluon $g$ is called the {\bf gluino} $\tilde g$, and as the gluon it is a colour octet Majorana fermion. We usually talk about the gluino as being one particle, however, as an adjoint representation of $SU(3)$ there are actually eight  (thus octet) distinct gluons, and we write $\tilde g^a$ when we want to make the distinction. As a colour octet it has nothing to mix with in the MSSM -- this is still true even if we allow for R-parity violation -- and at tree level the mass is given by the soft term $M_3$. Since it lives in the same superfield as the massless gluon it would otherwise had zero mass. 

The one complication for the gluino is that it is strongly interacting so $M_3(\mu)$ runs, relatively speaking, quickly with energy scale $\mu$. It is useful to instead talk about the scale-independent {\bf pole-mass}  $m_{\tilde{g}}$, meaning the pole of the renormalised propagator,
\[ \frac{i}{\slashed{p}-m_0-\Sigma(\slashed p)},\]
where $m_0$ is the Lagrangian mass, $\Sigma$ is the self-energy, and the pole mass is the solution $\slashed p=m$ to the equation $\slashed{p}-m_0-\Sigma(\slashed p) =0$.
For the gluino, including all one-loop effects in the self-energy due to gluon exchange and squark loops, see Fig.~\ref{fig:cogluion}, in the $\overline{DR}$ renormalisation scheme we get:\footnote{Note that the right-hand side here {\it is} $\mu$ dependent since the expression is only to finite order.} 
\[m_{\tilde{g}} \simeq M_3(\mu)\left[1 + \frac{\alpha_s}{4\pi}\left(15 + 6\ln\frac{\mu}{M_3}+ \sum_{{\rm all}\,\tilde q} A_{\tilde{q}}\right) \right],\]
where the squark loop contribution $A_{\tilde q}$ for a given squark depends on the squark $m_{\tilde{q}}$ and corresponding quark $m_q$ masses, and are given by
\[A_{\tilde{q}} = \int_0^1dx\,x \ln\left(x \frac{m_{\tilde{q}}^2}{M_3^2} +(1-x)\frac{m_q^2}{M_3^2} - x(1-x) -i\epsilon\right).\]
Due to the $15$-factor the correction can be significant (colour factor).
\begin{figure}[h]
\begin{center}
\includegraphics[scale=0.8]{figures/gluino.eps} 
   \caption{One loop contributions to the gluino mass. \label{fig:cogluion}}
\end{center}
\end{figure}

Complete Feynman rules for gluinos can be found in Appendix C of the classic MSSM reference paper of Haber \& Kane~\cite{Haber:1984rc}. A more comprehensible alternative may be Appendix A.3 from the PhD-thesis of Bolz~\cite{Bolz:2000xi}. This thesis also provides a diagramatic prescription of how to handle clashing fermion lines that can appear with Majorana fermions such as the gluino. 



%%%%%%%%%%%%%%%%%
\section{Neutralinos \& Charginos}
\label{sec:electroweakinos}
%%%%%%%%%%%%%%%%%
In the MSSM we have a bunch of fermion fields that can mix when electroweak symmetry is broken and we do not have to care about the $SU(2)_L\times U(1)_Y$ charges of the fields, only the charges under the remaining $U(1)_{\rm em}$ symmetry matter. The candidates for mixing are the (Majorana) fermions from the $U(1)$ and $SU(2)$ vector superfields $B^0$ and $W^a$ called the {\bf gauginos}:
\[\tilde{B}^0~\text{({\bf bino})}, \quad\tilde{W}^0~\text{(neutral {\bf wino})}, \quad \tilde{W}^\pm~\text{(charged {\bf wino})},\]
and the fermions from the Higgs superfields $H_u$ and $H_d$, called {\bf higgsinos}:
\[\tilde{H}^+_u, \quad \tilde{H}^0_u, \quad \tilde{H}^-_d \quad \text{and} \quad \tilde{H}^0_d.\]
Together these are called the {\bf electroweakinos}.

In the Standard Model the neutral gauge fields  $B^0_\mu$ and $W^0_\mu$ mix into the photon $\gamma$ and the $Z$-boson. The neutral gauginos can mix the same way into the states 
\begin{eqnarray}
\tilde{\gamma} &=& N'_{11}\tilde{B}^0 + N'_{12}\tilde{W}^0 \indent \text{({\bf photino})},\\
\tilde{Z} &=& N'_{21}\tilde{B}^0 + N'_{22}\tilde{W}^0 \indent \text{({\bf zino})}.
\end{eqnarray}
However, more generally, they also mix with the neutral higgsinos to form the four {\bf neutralinos}:\footnote{The neutral higgsinos are also Majorana fermions despite coming from scalar superfields. Unlike the (s)fermion superfields the Higgs superfields have no $\bar H$ chiral partners to supply the left--right Weyl spinor combinations required for Dirac fermions. Thus the neutralinos are Majorana fermions.}
\begin{equation}
\tilde{\chi}^0_i = N_{i1}\tilde{B}^0 + N_{i2}\tilde{W}^0 + N_{i3}\tilde{H}^0_d + N_{i4}\tilde{H}_u^0,\quad i=1,2,3,4,
\label{eq:neutralino}
\end{equation}
where $N_{ij}$ indicates size of the component of each of the fields in the {\bf gauge eigenstate basis}
\begin{equation}
\tilde{\psi}^{0T} = \begin{pmatrix} \tilde{B}^0,\,\tilde{W}^0,\,\tilde{H}^0_d,\, \tilde{H}_u^0\end{pmatrix}.
\label{eq:n_gauge_eigen}
\end{equation}

In the gauge eigenstate basis the neutralino mass term can be written as
\[\mathcal{L}_{\chi-{\rm mass}} = -\frac{1}{2}\tilde{\psi}^{0T}M_{\tilde\chi}\tilde{\psi}^0 + \text{c.c.},\]
where the mass matrix $M_{\tilde\chi}$ is found from the bilinear terms in the Lagrangian with gauge eigenstates to be
\begin{equation}
M_{\tilde\chi} =\begin{bmatrix}M_1 & 0 & -\frac{1}{\sqrt{2}}g'v_d & \frac{1}{\sqrt{2}}g'v_u\\ 0 & M_2 & \frac{1}{\sqrt{2}}gv_d & -\frac{1}{\sqrt{2}}gv_u\\ -\frac{1}{\sqrt{2}}g'v_d & \frac{1}{\sqrt{2}}gv_d & 0 & -\mu\\ \frac{1}{\sqrt{2}}g'v_u & -\frac{1}{\sqrt{2}}gv_u &-\mu & 0\end{bmatrix}.
\end{equation}
In this matrix, the upper left diagonal part comes from the soft terms for the $\tilde{B}^0$ and the $\tilde{W}^0$, the lower right off diagonal matrix comes from the superpotential term $\mu H_u H_d$, while the remaining entries come from Higgs-higgsino-gaugino terms from the kinetic part of the Lagrangian, {\it e.g.}\ $H_u^\dagger e^{\frac{1}{2}g\sigma W + g'B}H_u$, which become mass terms when one of the neutral Higgs component fields acquires a vev.
With the $Z$-mass condition on the vevs (\ref{eq:Zmass_vev_condition}) we can also write
\begin{eqnarray}
\frac{1}{\sqrt{2}}g'v_d &=& \cos\beta \sin\theta_W m_Z, \label{eq:gprime_vd} \\
\frac{1}{\sqrt{2}}g'v_u &=& \sin\beta \sin\theta_W m_Z, \\
\frac{1}{\sqrt{2}}gv_d &=& \cos\beta \cos\theta_W m_Z, \\
\frac{1}{\sqrt{2}}gv_u &=& \sin\beta \cos\theta_W m_Z .\label{eq:g_vu}
\end{eqnarray}

The mass matrix can now be diagonalised to find the $\tilde\chi^0_i$ masses. If $N$ is a unitary diagonalisation matrix for $M_{\tilde\chi}$, we can write\footnote{A symmetric matrix is always unitary diagonalisable.}
\[\mathcal{L}_{\chi-{\rm mass}} = -\frac{1}{2}\tilde{\psi}^{0T}N^TN^*M_{\tilde\chi} N^\dagger N \tilde{\psi}^0 + \text{c.c.},
= -\frac{1}{2}\tilde{\chi}^{0T} D\tilde{\chi}^0 + \text{c.c.},\]
where $D=N^*M_{\tilde{\chi}}N^\dagger$ is diagonal and contains the real and non-negative neutralino masses $m_{\tilde\chi_i^0}\ge0$ that are the eigenvalues of $M_{\tilde\chi}$. We also see that $N$ gives the mixing of the gauge eigenstates  $\tilde{\psi}^0$ into the mass eigenstates $\tilde\chi^0=N\tilde{\psi}^0$. We number the neutralino mass eigenstates in (\ref{eq:neutralino}), or, equivalently, sort the mass eigenvalues after diagonalisation, so that the neutralinos are numbered from lightest to heaviest. The neutralinos also have loop corrections to their masses coming from the self energies, however, since the coupling is weak -- in the technical sense -- these are usually significantly smaller than for the gluino.


The mass parameters of the neutralino mass matrix may in general be complex, leading to complex entries in $N$. Redefinition of fields can rotate away either the $M_1$ or $M_2$ phase, to make the parameter real and positive, but not both of these, and not the $\mu$-phase. These phases give rise to problematic CP-violation that can easily be in contradiction with experiments. Therefore, $M_1$, $M_2$ and $\mu$ are often just assumed to be real in order not to violate experimental bounds. In this case a diagonalisation matrix $N$ can be found that is orthogonal, meaning with only real entries, which simplifies calculations. In this case the diagonal mass values in $D$ are not guaranteed to be positive. This does not imply negative fermion masses, but instead indicates a phase factor that must be incorporated into Feynman rules for the interactions of the mass eigenstates.

One particularly interesting solution to the diagonalisation is in the limit where EWSB is a small effect, $m_Z\ll |\mu \pm M_1|$, $|\mu \pm M_2|$, and when we have the hierarchy $M_1<M_2\ll |\mu|$. The mass eigenvalues scale with the size of the supersymmetric parameters, which makes the lightest neutralino bino-like, $\tilde{\chi}_1^0\approx \tilde{B}^0$, the next-to-lightest wino like, $\tilde{\chi}_2^0\approx \tilde{W}^0$, and $\tilde{\chi}^0_{3,4} \approx \frac{1}{\sqrt{2}}(\tilde{H}^0_d \pm \tilde{H}^0_u)$, and the masses are to first order in $1/\mu$:
\begin{eqnarray}
m_{\tilde{\chi}^0_1} &=& M_1 +\frac{m_Z^2\sin^2\theta_W\sin2\beta}{\mu} +\ldots\\
m_{\tilde{\chi}^0_2} &=& M_2 -\frac{m_W^2\sin2\beta}{\mu} +\ldots\\
m_{\tilde{\chi}^0_{3,4}} &=& |\mu| +\frac{m_Z^2}{2\mu}({\rm sgn}\,\mu \mp \sin2\beta) + \ldots
\end{eqnarray}

Since the LSP is stable in R-parity conserving theories the lightest neutralino is an excellent candidate for dark matter. In particular since a neutralino with mass around 100 GeV has a natural relic density close to the measured dark matter density of the Universe. We will return to this issue in Chapter~\ref{chap:dm}.

From the charged electroweakinos we can make {\bf charginos} $\tilde{\chi}^{\pm}_i$  that are Dirac fermions with mass term
\[\mathcal{L}_{\chi^\pm-{\rm mass}} = -\frac{1}{2}\tilde{\psi}^\pm{}^TM_{\chi^\pm}\tilde{\psi}^{\pm} + \text{c.c.},\]
where the gauge eigenstate basis is $\tilde{\psi}^{\pm T} = (\tilde{W}^+,\, \tilde{H}^+_u ,\,\tilde{W}^-,\, \tilde{H}^-_d)$, and the mass matrix is given by
\[M_{\tilde{\chi}^\pm} =\begin{bmatrix}0 & 0 & M_2 & gv_d\\ 0 &0 & gv_u & \mu\\ M_2 & gv_u & 0 & 0 \\ gv_d &\mu & 0&0\end{bmatrix}.\]
Here the $M_2$ terms come from the soft terms for the charged winos $\tilde W^\pm$, the $\mu$ terms come from the superpotential as above, while the remaining terms come from the kinetic terms. We can here re-write
\begin{eqnarray}
gv_d &=& \sqrt{2}\cos\beta\, m_W,\\
gv_u &=& \sqrt{2}\sin\beta\, m_W.
\end{eqnarray}

Diagonalising this mass matrix gives the mass eigenstates $\tilde\chi_i^\pm$, $i=1,2$. The eigenvalues are doubly degenerate, giving the same masses to the $\tilde\chi_i^+$ and $\tilde\chi_i^-$ particle and anti-particle pairs, and are explicitly given as:
\[m_{\tilde{\chi}^\pm_{1,2}} = \frac{1}{2}\left(|M_2|^2 + |\mu|^2 + 2m_W^2 \mp \sqrt{(|M_2|^2 + |\mu|^2 + 2m_W^2)^2 - 4|\mu M_2-m_W^2\sin^2\beta|^2}\right).\]

In the same limit of small EWSB effects discussed above we have a wino-like lightest chargino, $\tilde{\chi}^\pm_1 \approx \tilde{W}^\pm$, and a higgsino-like heavy chargino, $\tilde{\chi}^\pm_2 \approx \tilde{H}^+_u/\tilde{H}^-_d$, with masses 
\begin{eqnarray}
m_{\tilde{\chi}^\pm_1} &=& M_2 - \frac{m_W^2}{\mu}\sin2\beta+\ldots,\\
m_{\tilde{\chi}^\pm_2} &=& |\mu| + \frac{m_W^2}{\mu}{\rm sgn}\,\mu+\ldots.
\end{eqnarray}
Note that in this limit $m_{\tilde{\chi}^0_2} \simeq m_{\tilde{\chi}^\pm_1}$ since they are both wino-like and governed by the $M_2$ soft mass.

We saw earlier that the soft-mass ratio
\[M_3:M_2:M_1 \simeq 6:2:1,\]
appears at a scale of around $\mu=1$\,TeV if the same soft-masses unify at the GUT-scale. From our above discussion, as long as $|\mu|\gg M_1,M_2$, this gives the very predictive mass relationships $m_{\tilde{g}} \simeq 6m_{\tilde{\chi}^0_1}$, $m_{\tilde{\chi}^0_2} \simeq m_{\tilde{\chi}^\pm_1} \simeq 2m_{\tilde{\chi}^0_1}$. However, it is important to remember that this often used relationship is based on the {\it conjecture} of gauge coupling unification, and the unification of gaugino soft masses!


We should mention that some authors prefer other symbols for the neutralinos and charginos. Common examples are $\tilde N_i$ or $\tilde Z_i$ for neutralinos, and $\tilde C_i$ or $\tilde W_i$ (again!) for the charginos. 

Feynman rules for charginos \& neutralinos can again be found in Haber \& Kane~\cite{Haber:1984rc}.




%%%%%%%%%%%%%%%
\section{Sleptons \& Squarks}
%%%%%%%%%%%%%%%
The {\bf sfermions}, the scalar partners of the Standard Model fermions, the quarks and leptons, consists of the {\bf squarks} and the {\bf sleptons}. These inherit the interactions of their partner fermions since they live in the same superfields.

For their masses, reading of from the MSSM Lagrangian, including the possible soft-breaking terms,  there are multiple tree-level contributions to the sfermion masses. In the following discussion  $\tilde{F}_i$ represents a generic $SU(2)_L$ doublet of sfermions with generation index $i$, for example $\tilde Q_i=(\tilde u_{iL},\tilde d_{iL})$, while $\tilde{f}_{iR}$ represents a singlet, for example $\tilde u_{iR}$.

We can make the following list of mass terms: 
\begin{enumerate}[i)]
\item Under the reasonable assumption that soft masses are (close to) diagonal\footnote{This assumption is of course made to avoid flavour changing neutral currents (FCNCs). However, it is also reasonable in that if the soft masses are diagonal, or even all the same, at a high scale, the RGE running will not create large off-diagonal terms. } the sfermions get contributions $-m_{F_i}^2\tilde{F}_i^\dagger \tilde{F}_i$ and $-m_{f_i}^2\tilde{f}^*_{iR}\tilde{f}_{iR}$ from the soft terms in (\ref{eq:MSSM_soft_kin}). These are typically dominant.
\item There are $F$-term contributions that come from Yukawa terms in the superpotential of the form $y_fFH\bar K$, where $F$ and $\bar K$ are two scalar superfields with sfermions, and $H$ is one of the two Higgs superfields. From the contribution $\sum |W_i|^2$ to the scalar potential
these give Lagrangian terms $y_f^2H^0{}^*H^0\tilde{f}^*_{iL}\tilde{f}_{iL}$ and $y_f^2 H^{0*}H^0\tilde{f}^*_{iR}\tilde{f}_{iR}$. After EWSB when the Higgs field gets a vev we then get the mass terms $m_f^2 \tilde{f}^*_{iL}\tilde{f}_{iL}$ and $m_f^2\tilde{f}^*_{iR}\tilde{f}_{iR}$, where $m_f = v_{u/d}\,y_f$. These are only significant for large Yukawa coupling $y_f$, and give the same mass as their Standard Model fermion partner gets from the same Yukawa terms.
\item There are also so-called {\bf hyperfine} terms that come from $D$-terms $\sum g^2(A^*T^aA)^2$ in the scalar potential that give Lagrangian terms of the form (sfermion)$^2$(Higgs)$^2$ when one of the scalar fields $A$ is a neutral Higgs field, and the other is a sfermion. Under EWSB, when the Higgs field gets a vev these become mass terms. They contribute with a mass \[\Delta_F = (T_{3F}g^2 - Y_Fg'{}^2)(v_d^2-v_u^2) = (T_{3F} - Q_F\sin^2\theta_W)\cos2\beta\, m_Z^2,\] where the weak isospin, $T_3$, hypercharge, $Y$, and electric charge, $Q$, are for the left-handed supermultiplet $F$ to which the sfermion belongs. However, these contributions are usually quite small.
\item Furthermore, there are also $F$-terms that combine scalars from the $\mu H_uH_d$ term and Yukawa terms $y_fFH\bar K$ in the superpotential. These give Lagrangian terms $-\mu^*H^0{}^* y_f \tilde{f}_L \tilde{f}_R^*$. With a Higgs vev this gives mass terms $-\mu^* v_{u/d}\,y_f\tilde{f}^*_R\tilde{f}{}_L + {\rm c.c.}$
\item Finally, the soft Yukawa terms of the form $a_f \tilde{F}H\tilde{f}^*_R$ with a Higgs vev give mass terms $a_f v_{u/d} \tilde{f}_L\tilde{f}^*_R + {\rm c.c.}$\footnote{We often assume that $a_f = A_0 y_f$ in order to further reduce the FCNC, meaning that there is a global constant $A_0$ with unit mass relating the Yukawa couplings and the trilinear A-term couplings.} 
\end{enumerate}

For the first two generations of sfermions, terms of type ii), iv) and v) are small due to small Yukawa couplings. Then the sfermion masses are for example
\begin{eqnarray}
m_{\tilde{u}_L}^2 &=& m^2_{Q_1} + \Delta \tilde{u}_L,\\
m_{\tilde{d}_L}^2 &=& m^2_{Q_1} + \Delta \tilde{d}_L,\\
m_{\tilde{c}_L}^2 &=& m^2_{Q_2} + \Delta \tilde{c}_L,\\
m_{\tilde{s}_L}^2 &=& m^2_{Q_2} + \Delta \tilde{s}_L,\\
m_{\tilde{u}_R}^2 &=& m_{u_1}^2 + \Delta \tilde{u}_R \\
m_{\tilde{d}_R}^2 &=& m_{d_1}^2 + \Delta \tilde{d}_R \\
m_{\tilde{s}_R}^2 &=& m_{u_2}^2 + \Delta \tilde{u}_R.
\end{eqnarray}
Mass splitting between same generation slepton/squark is then given by the hyperfine splitting  \[m_{\tilde{e}_L}^2 - m_{\tilde{\nu}_{eL}}^2 = m_{\tilde{d}_L}^2-m_{\tilde{u}_L}^2  = -\frac{1}{2}g^2 (v_d^2 - v_u^2) = -\cos 2\beta\, m_W^2,\] since they have the same hypercharge, see Table~\ref{tab:hyper}. For $\tan\beta >1$ this gives the definite prediction $m_{\tilde{e}_L}^2 > m_{\tilde{\nu}_{eL}}^2$ and $ m_{\tilde{d}_L}^2 > m_{\tilde{u}_L}^2$.

The {\bf third generation sfermions} $\tilde{t}$, $\tilde{b}$ and $\tilde{\tau}$ have a more complicated mass matrix structure, {\it e.g.}\ in the gauge eigenstate basis $(\tilde{t}_L,\, \tilde{t}_R)$ for stop quarks the mass term is
\[\mathcal{L}_{\rm stop} = -\begin{pmatrix}\tilde{t}_L^*& \tilde{t}_R^*\end{pmatrix}m^2_{\tilde{t}}\begin{pmatrix}\tilde{t}_L\\ \tilde{t}_R\end{pmatrix},\]
where the mass matrix is given by
\begin{equation}
m_{\tilde{t}}^2 = \begin{bmatrix}m_{Q_3}^2 + m_t^2 + \Delta \tilde{u}_L & v(a_t^*\sin\beta - \mu y_t \cos\beta)\\ v(a_t\sin\beta - \mu^* y_t \cos\beta) & m_{u_3}^2 + m_t^2 + \Delta \tilde{u}_R \end{bmatrix}.
\label{eq:stopmassmatrix}
\end{equation}
Here the diagonal elements come from i), ii) and iii), while the off-diagonal elements come from iv) and v). 

To find the particle masses, we must diagonalise this matrix, writing it in terms of the mass eigenstates $\tilde{t}_1$ and $\tilde{t}_2$,  acquiring also here  a unitary mixing matrix for the mass eigenstates in terms of the gauge eigenstates $\tilde{t}_L$ and $\tilde{t}_R$:
\begin{equation}
\begin{pmatrix}\tilde{t}_1\\ \tilde{t}_2\end{pmatrix} 
=\begin{bmatrix}c_{\tilde t} & -s_{\tilde t}^* \\ s_{\tilde t} & c_{\tilde t} \end{bmatrix}
\begin{pmatrix}\tilde{t}_L\\ \tilde{t}_R\end{pmatrix},
\end{equation}
where the matrix entries are related by $|c_{\tilde t} |^2+|s_{\tilde t} |^2=1$ and  $m_{\tilde t_1}^2<m_{\tilde t_2}^2$ are the eigenvalues of (\ref{eq:stopmassmatrix}). The suggestive form of the mixing matrix indicates that if the off-diagonal elements of the original mass matrix has only real elements, this mixing matrix can be written as an element in $SO(2)$, using sine and cosine of a mixing angle $0\le\theta_{\tilde t}<\pi$,  $c_{\tilde t}=\cos\theta_{\tilde t}$ and $s_{\tilde t}=\sin\theta_{\tilde t}$.
The matrices for $\tilde{b}$ and $\tilde{t}$ have the same structure. 

Since the third generation sneutrino $\tilde\nu_{eL}$ does not have a corresponding right-handed state in the MSSM, there is no mixing, and it has the same mass term as the first and second generation sneutrinos.

A good source for sfermion interaction Feynman rules is the PhD-thesis of Richardson~\cite{Richardson:2000nt}.



%%%%%%%%%%%%%
\section{Excercises}
%%%%%%%%%%%%%

\begin{Exercise}[]
Using the explicit form of the $SU(3)_C$ transformations with the Gell-Mann matrices, show that with our definition of the superpotential term $\bar{U}_i\bar{D}_j\bar{D}_k$ this is invariant under $SU(3)_C$.
\end{Exercise}


\begin{Exercise}[]
Show how you can eliminate the parameters $|\mu|$ and $b$ by using the properties of the minimum of the potential in Eq.~(\ref{eq:higgspot_gauged}).
\end{Exercise}

\begin{Exercise}[]
Show Eqs.~(\ref{eq:gprime_vd})--(\ref{eq:g_vu}).
\end{Exercise}

\end{document}



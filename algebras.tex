% Standalone document
\documentclass[notes.tex]{subfiles}
\begin{document}
%%%%%%%%%%%%%%%%%%%%%%%%%%%%%%%%%%%%%%%%%%%%%%%%%%%%%%%
%%%%%%%%%%%%%%%%%%%%%%%%%%%%%%%%%%%%%%%%%%%%%%%%%%%%%%%
\chapter{Algebras}
\label{chap:algebras}
%%%%%%%%%%%%%%%%%%%%%%%%%%%%%%%%%%%%%%%%%%%%%%%%%%%%%%%
%%%%%%%%%%%%%%%%%%%%%%%%%%%%%%%%%%%%%%%%%%%%%%%%%%%%%%%

%%%%%%%%%%%
\section{Properties of algebras}
\label{sec:algebras}
%%%%%%%%%%%

To further study the structure of groups we begin by defining an {\bf algebra}. An algebra extends the familiar structure of vector spaces by adding a multiplication operation for the vectors which gives a new vector.
\df{An {\bf algebra} $A$ over a field (say $\mathbb{R}$ or $\mathbb{C}$) is a linear vector space with a binary (multiplication) operation $\circ : A \times A \to A$.}
It is important to remember here that the vector space part of the definition implies that there is field, so for example from $\mathbf x\in A$ and $a\in \mathbb{R}$ (as the field) we can always form new members $a\mathbf x\in A$.

As a very simple example, the vector space $\mathbb{R}^3$ together with the standard cross-product constitutes an algebra since the cross product results in a new vector in  $\mathbb{R}^3$. Even more trivially perhaps is that $\mathbb{R}$ with ordinary multiplication as the binary operation fulfils the algebra requirements (more on this case below).

Connecting to the discussion of Lie groups in Section \ref{sec:Lie_groups}, we see that the generators $X_i$ of a Lie group also span an algebra as basis vectors of a vector space, if we use the commutator as  the multiplication operation between two members. We leave it as an exercise to ponder out how all the criteria for a vector space is fulfilled by the generators, but this is not very difficult.


%%%%%%%%%%%
\section{Normed division algebras$^*$}
\label{sec:normed_division_algebras}
%%%%%%%%%%%
As a slight aside to the main argument of the text, we want to give an interesting example of algebras. We start with a {\bf division algebra}, which informally is an algebra where the binary operation of the algebra also has a meaningful (implicit) concept of division for all members (except division by zero).
\df{An algebra $D$ is a {\bf division algebra} if for any element $x$ in $D$ and any non-zero element $y$ in $D$ there exists precisely one element $z$ in $D$ with $x = y\circ z$ and precisely one element $z'$ in $D$ such that $x = z'\circ y$. In this sense $y$ is a divisor of $x$.}

We see that division algebras have the addition and subtraction properties of ``ordinary numbers'' (reals) since they are vector spaces, and they have the  multiplication (algebra) and division (division algebra) properties of the reals as well. So, in a sense, division algebras are structures close to the reals in terms of properties -- and, of course, the reals are again an example of a division algebra. 

We can now add to the division algebra the notion of a {\bf norm}, or length, of the members $\|x\|$. This is a map from the algebra to the field, $\|\,\|:D\to\mathbb{R}$, so that we can discuss for example convergence of the members as we do for the reals. We have to require here that the norm is homomorphic, meaning that it preserves the structure of the algebra, so that for example the ``product''  of two objects with large norm has a large norm. We then get a {\bf normed division algebra}. 
\df{If there exists a homomorphic norm for the division algebra, {\it i.e.}\ one where $\|x\circ y\|=\|x\|\|y\|$ for all $x,y\in D$, then the division algebra is a {\bf normed division algebra}.}

There is an important theorem by Hurwitz~\cite{Hurwitz1923} that demonstrates that only four of these real number ``lookalikes'' exist.
\theo{Hurwitz's theorem. There are only four normed division algebras over the reals (up to isomorphism), the reals themselves $\mathbb{R}$, the complex numbers $\mathbb{C}$,  the quaternions $\mathbb{H}$, and the octonions $\mathbb{O}$.}

In addition it is (relatively speaking) easy to show that $\mathbb{R} \subset\mathbb{C}\subset\mathbb{H}\subset\mathbb{O}$. One perspective on the relationship between these algebras is that the reals is the only ordered normed division algebra, {\it i.e.}, where we can compare uniquely the elements $a>b$. This is not possible for the complex numbers, but they keep the commutative and associate properties of the reals. The quaternions in turn break the commutativity of the complex numbers, while being associative, while the most unruly of the bunch, the octonions, are not even associative.


%%%%%%%%%%%
\section{Lie algebras}
\label{sec:lie_algebras}
%%%%%%%%%%%
We will now turn to the most crucial type of algebras for physics, namely {\bf Lie algebras}. To distinguish these from the more general algebras that we have introduced above, we will use the notation $[\,,\,]$ for the binary operation in Lie algebras. 
\df{A {\bf Lie algebra} $\mathfrak l$  is an algebra where the binary operator $[\,,\,]$, called the {\bf Lie bracket}, has the properties that for $x,y,z \in \mathfrak l$ and $a,b \in \mathbb{R}$ (or $\mathbb{C}$):
\begin{enumerate}[i)]
\item (bilinearity)
 \[[ax + by, z] = a[x,z] + b[y,z]\]
\[[z, ax + by] = a[z,x] + b[z,y]\]
\item (anti-commutation) \[[x,y] = -[y,x] \]
\item (Jacobi identity) \[[x, [y, z]] + [y, [z, x]] + [z, [x, y]] = 0\]
\end{enumerate}}
If the algebra is over $\mathbb R$ it is a real Lie algebra, and over $\mathbb C$ a complex Lie algebra. We write the name of Lie algebras $\mathfrak l$  in lower case {\bf fraktur} style. A Lie algebra is called {\bf abelian} if its Lie bracket is always zero.

Since Lie algebras are vector spaces, a {\bf subalgebra} $\mathfrak m$  of a Lie algebra $\mathfrak l$ is simply a subspace $\mathfrak m\subseteq \mathfrak l$ that is closed under the Lie bracket. An {\bf ideal} is a subalgebra that also satisfies $[\mathfrak l, \mathfrak m] \subseteq \mathfrak m$, where we interpret the notation as saying that the product by Lie bracket of elements in $\mathfrak l$ and $\mathfrak m$ are in $\mathfrak m$.   A {\bf simple} Lie algebra is a non-abelian Lie algebra without any proper ideals. A Lie algebra is {\bf semi-simple} if it is a direct sum of simple Lie algebras.

Again the vectors of $\mathbb{R}^3$ with the Lie bracket defined in terms of the cross product, $[\mathbf x, \mathbf y]=\mathbf x \times \mathbf y$, is a simple example of a Lie algebra. However, we usually restrict ourselves to algebras of linear operators where the Lie bracket is the standard commutator $[x,y] \equiv xy-yx$, where the defining properties follow automatically. Thus also explaining the notation that we have used for the binary operator. 

We have the same notion of sameness for the Lie algebras as for the groups, namely that of isomorphism. A Lie algebra {\bf homomorphism} is a linear map 
$\rho:\mathfrak l \to \mathfrak m$ where
\[ \rho ( [ x , y ] ) = [ \rho ( x ) , \rho( y ) ], \]
for all $x,y\in \mathfrak l$, and an {\bf isomorphism} of Lie algebras is a one-to-one and onto homomorphism. 

From what we learnt about structure constants  in Section \ref{sec:Lie_groups} the generators $X_i$ of an $d$-dimensional Lie group then span a $d$-dimensional Lie algebra with the commutator as the Lie bracket.\footnote{If you are really paranoid here, you may question whether these generators are guaranteed to be linearly independent. However, since the Lie group is parameterised by $d$ parameters, if the generators were not linearly independent we should be able to remove one of the parameters, leading to a contradiction. Stop it!} Since this is also a vector space, any element $X$ of the algebra can be written in terms of the generators $X_i$, as $X=a_iX^i$. This is called the {\bf tangent space} of the group at the identity. The differences between choices in the definition of the generators and the normalisation of the structure constants disappear as they can be absorbed into a rescaling and linear combinations of the basis $\{X_i\}_{i=1}^d$, this makes the Lie algebra of the group {\it unique}. 
%Given a Lie group, the Jacobi identity for its Lie algebra follows from the associativity of the group operation. 
In the correspondence between Lie groups and Lie algebras, subgroups correspond to Lie subalgebras, and normal subgroups correspond to ideals. 

However, the reverse is not true. There can be multiple Lie groups that have the same algebra. The often quoted example is the groups $SO(3)$ and $SU(2)$ where we have seen that the commutator of the generators is the same up to constants, so the algebras are isomorphic, we say $\mathfrak{so(3)}\cong \mathfrak{su(2)}$. However, while closely related, the groups are not isomorphic. 


%%%
\subsection{Representations of a Lie algebra}
\label{sec:algebra_rep}

We saw in the previous subsection that -- just as for groups -- we have many different way of writing down a single Lie algebra. We then need a formal definition of representations of Lie algebras.
\df{A {\bf representation} $\pi$ {\bf of a Lie algebra} $\mathfrak l$ on a vector space $V$  is a map
\[\pi: \mathfrak{l} \to \mathfrak{gl}(V), \]
where $ \mathfrak{gl}$ is the Lie algebra of all linear maps $V\to V$ (endomorphisms) with its bracket defined as the composition $[r,s]\equiv r\circ s-s\circ r$, for all $r,s\in \mathfrak{gl}$, and where the map $\pi$ is homomorphic (structure preserving) for the Lie algebra bracket so that for all $x,y\in \mathfrak{l} $,
\[ \pi([x,y]) =\pi(x)\circ\pi(y)-\pi(y)\circ\pi(x). \]
}
Again this is an abstract definition that is actually simpler than it looks. For finite dimensional vector spaces $\mathfrak{gl}$ is the Lie algebra of the corresponding general linear matrix group, just as the definition of representations of groups was a map to the general linear group $GL(n)$. This algebra of $GL(n)$, called $\mathfrak{gl}(n)$, can be written up in terms of $n\times n$ matrices and their commutators, and in this case the multiplication operation $\circ$ is just matrix multiplication in hiding. One word of warning here though, while the group $GL(n)$ contained invertible matrices, there is no guarantee that its generators are invertible matrices, {\it e.g.} we saw earlier that the generators of $SO(3)$ were the matrices $L_i$ in Eq~(\ref{eq:SO3_generators}) that are not invertible since $\det L_i=0$.

What makes this definition of representations even simpler for Lie algebras is that {\bf Ado's theorem}~\cite{Ado:1935} proves that every finite-dimensional Lie algebra has a faithful (isomorphic to the Lie algebra) representation on a finite-dimensional vector space.\footnote{For a modern version of a proof of Ado's theorem, see Terence Tao's blog \url{https://terrytao.wordpress.com/2011/05/10/ados-theorem/}.} In other words, the Lie algebras for the groups we most care about, Lie groups with a finite number of parameters and thus finite number of generators, is guaranteed to have representations using matrices.

As for groups, we can talk about irreducible representations, irreps, of the Lie algebras, which are representations such that $V$ has no proper invariant subspace under the representation, meaning that we cannot construct proper sub-representations.
% Maybe say something about similarity transforms between different Lie algebras 
An important result for representations of Lie groups is {\bf Weyl's complete reducibility theorem} which says that the any representation of a finite-dimensional semi-simple Lie algebra  on a finite-dimensional space is isomorphic to a direct sum of irreducible representations. This means that the classification of the representations of semi-simple Lie algebras can be done in a very systematic way. In particular, every semi-simple Lie algebra is a subalgebra of $\mathfrak {sl}$, the Lie algebra of the special linear group, and all our matrix groups have simple or semi-simple Lie algebras.
%TODO: extend to semi-simple representation theory here

In Sec.~\ref{sec:generators} we saw the two sets of generators of the Lie algebras $\mathfrak{su(2)}$ and $\mathfrak{so(3)}$, derived from the defining matrix representations of the groups that have different dimensions. Since this is the same algebra this shows two distinct representations of the same algebra. 


%%%
\subsection{Exponential map}
\label{sec:expmap}
As we discussed in Section \ref{sec:Lie_groups}, the generators describe the local structure of the group near the identity element. We can now finally look at how the group (and matrix representation) is reconstructed from the algebra. For this we use what is called the {\bf exponential map}. 
\df{The {\bf exponential map} from a finite-dimensional Lie algebra $\mathfrak l$ of a group $G$ to $G$ is defined by $\exp:\mathfrak l\to G$, where for $X\in\mathfrak l$ we get the element $g\in G$ given by
\begin{equation}
g=\exp(iX)\equiv\sum_{n=0}^{\infty}\frac{(ia_iX^i)^n}{n!}.
\end{equation}
Here the $X_i$ are the generators of $G$ and thus members of the Lie algebra, and $X=a_iX^i$.
}
By Ado's theorem any (finite-dimensional) Lie algebra can be represented by matrices, in which case the infinite sum in this definition is nothing more than the formal series definition of the matrix exponential introduced in Sec.~\ref{sec:unitary_groups}. However, we can also use the exponential map for differential representations where the $X_i$ are differential operators, and for infinite-dimensional Lie groups, but the technical aspects why this extension works is a little beyond the scope of these notes.

We have already seen multiple example of the exponential map construction, but we may ask, why use the exponential? One reason is that it gives us a natural expansion of the group around the identity element. With the parameters $\mathbf a=0$ the resulting matrix is the identity matrix, which means that the group element is the identity element $g=e$. Similarly, for an infinitesimal $d\mathbf a$ step away from the parameters of the identity we have the group elements near the identity, $g=I+ida_iX^i$ that we used to derive the generators. And now, if we apply a small finite step  $a_i$ away from the identity $g=I+i\frac{a_i}{n}X^i$ a total of $n$ times we will get
\[g=\left(I+i\frac{a_i}{n}X^i\right)^n. \]
By the limit definition of the exponential, with smaller and smaller steps letting $n\to\infty$ this is the exponential map
\[g=\lim_{n\to\infty}\left(I+i\frac{a_i}{n}X^i\right)^n=\exp( ia_iX^i). \]

We should now ask two important questions, is the structure of these elements $g$ actually a group, and is it (isomorphic to) the group that gave the Lie algebra? For the first question we may check the group definition. The exponential map does give an identity element which is just the identity matrix. Associativity similarly holds because we are using matrices and matrix multiplication. There is indeed an inverse of an element $g=\exp{(iX)}$, which is $g^{-1}=\exp{(-iX)}$, since by the properties of matrix exponentiation we have $gg^{-1}=\exp{(iX)}\exp{(-iX)}=I$. The tricky point is whether group multiplication is closed.

If we have two elements from the map $g=\exp{(iX)}$ and $g'=\exp{(iY)}$, then $gg'=\exp{(iX)}\exp{(iY)}$, and we would like to show that  $\exp{(iX)}\exp{(iY)}=\exp{(iZ)}$, where $Z\in\mathfrak l$. However, unless $X$ and $Y$ commute we cannot  easily join them in the same exponential. We here use the {\bf Baker--Campbell--Haussdorff  formula},
\begin{equation}
Z=X+Y+\frac{1}{2}[X,Y]+\frac{1}{12}[[X,Y],Y]-\frac{1}{12}[[X,Y],X]+\ldots,
\label{eq:BKH_matrices}
\end{equation}
where the omitted terms involve Lie brackets of four or more elements, and where every element in the series contains the commutator $[X,Y]$. If $X$ and $Y$ are close enough to the zero in $\mathfrak l$ (corresponding to the identity element in $G$) then this series can be shown to converge to an element in $\mathfrak l$.\footnote{This formula holds even if the elements in the Lie algebra ${\mathfrak l}$ are not represented as finite-dimensional matrices.} Thus the exponential constructs a group at least locally around the identity element.

The question of whether the exponential map reaches all of the members of the group, {\it i.e.}\ that it is surjective on $G$, depends on the properties of the group. We certainly have counter examples, since the groups $SO(3)$ and $SU(2)$ have the same Lie algebra and thus the same exponential map, but are different groups, the map cannot reach all of the elements of both groups. What we do know is that locally, meaning sufficiently close to the identity group element, the exponential map generates the group.

Let us end this subsection by returning to some of our examples in view of what we now know about the exponential map. For $U(1)$ we saw that we could write a generic group member as $e^{i\theta}$. Comparing to the exponential map we see that the single generator must here simply be 1, while $\theta$ is the parameter. The Lie algebra is then a very simple one-dimensional vector space $\mathbb R$ where the multiplication is ordinary arithmetic multiplication. Similarly, for the translation group $T(1)$ we saw that a group member could be written as $e^{a\frac{\partial}{\partial x}}=e^{ia(-i\frac{\partial}{\partial x})}$. Thus the generator is the differential operator $P=-i\frac{\partial}{\partial x}$ and again the vector space of the Lie algebra is one dimensional.

For $SU(2)$ the generators taken from the fundamental representation $\mathbf 2$ were proportional to the Pauli matrices $X_i= \frac{1}{2}\sigma_i$, and the exponential map generating a group member $U$ is thus, as we alluded to earlier, $U=e^{i\alpha_iX^i}$, where $\alpha_i$ are the parameters. For $SO(3)$ the (Hermitian) generators taken from the fundamental representation $\mathbf 3$ are the matrices $Y_i=-iL_i$, where the $L_i$ were given in (\ref{eq:SO3_generators}), and the exponential map is $O=e^{i\alpha_iY^i}$.
Since $SU(2)$ and  $SO(3)$ have the same algebra for $X_i$ and $Y_i$, but the groups are {\it not} isomorphic,  in what sense, if any, does these exponential maps generate different elements? The answer to that is that $SU(2)$ is a {\bf double cover} of $SO(3)$, it has double the number of elements. There exists a two-to-one group homomorphism $f:SU(2)\to SO(3)$. This can be represented by $U(\alpha_1,\alpha_2,\alpha_3)\to O(\alpha_1,\alpha_2,\alpha_3)$. To see that this is two-to-one, assume that $\alpha_1=\alpha_2=0$. Then $O(0,0,\alpha_3)=O(0,0,\alpha_3+2\pi)$ since a rotation by an extra $2\pi$ does nothing. For $SU(2)$ we instead have
\[
U(0,0,\alpha_3)=\left[\begin{matrix} e^{i\frac{\alpha_3}{2}} & 0 \\ 0 & -e^{i\frac{\alpha_3}{2}} \end{matrix}\right] \quad \text{and}\quad U(0,0,\alpha_3+2\pi)=\left[\begin{matrix} e^{i\frac{\alpha_3}{2}+i\pi} & 0 \\ 0 & -e^{i\frac{\alpha_3}{2}+i\pi} \end{matrix}\right]=-U(0,0,\alpha_3)
\]
Thus the $SU(2)$ matrices has a period $4\pi$ versus the period $2\pi$ for $SO(3)$.\footnote{The argument can be made more formal and independent of the parameterisation, but is to extensive to repeat here.} Notice that this means that the three-dimensional representation of the Lie algebra coming from the generators of $SO(3)$ does not generate all the elements of $SU(2)$, while the two-dimensional one does.
% TODO: Better description of double cover in 3.2 https://www.weizmann.ac.il/particle/perez/Courses/QMII18/TA7.pdf

Beyond the fundamental representation $SU(2)$ has irreducible representations in all dimensions $\mathbf 3$, $\mathbf 4$, $\mathbf 5$, \ldots. This is not true in general, for example $SU(3)$ has irreducible representations only in some dimensions, starting with $\mathbf 3$, $\mathbf 6$, $\mathbf 8$, and $\mathbf{10}$, \ldots.

From the exponential map and the properties of matrix exponentiation it immediately follows that the generators $T^a$ for all groups $SU(n)$ are matrices with zero trace. To see this write an element in $SU(n)$ as $U=\exp{(i\alpha_aT^a)}$, then by the defining group property $\det U=1$, we have $\det U=\exp{(\Tr[{i\alpha_aT^a}}])=\exp{(i\alpha_a\Tr{T^a}})$ which is only one if $\Tr T^a=0$ for all $a$.  

We saw in Sec.~\ref{sec:unitary_groups} that we could construct unitary matrices from Hermitian ones by matrix exponentiation, but in fact for unitary groups the generators need to be Hermitian. To see this let a group member be $U=\exp(ia_iM^i)$. Then $U^\dagger = \exp(ia_iM^i)^\dagger=\exp(-ia_iM^{i\dagger})$,  but for a unitary matrix $U^\dagger = U^{-1}=\exp(-ia_iM^i)$, thus $M^{i\dagger}=M^i$ and the generators $M_i$ are Hermitian.
%TODO: Add tracelessness for $SU$. If $U=\exp(ia_iM^i)$ then $\det U = \exp{ia_i\Tr[M^i]}$ and if $\det U =1$ then we must have $\Tr M_i =0$ for all $i$.


%%%
\subsection{Adjoint representations}
\label{sec:adjoint_reps}
The definition of the structure constants in Sec.~\ref{sec:structure_constants} allows us to introduce another representation, the {\bf adjoint representation} of the algebra -- and by the exponential map the adjoint representation of the group -- where the representation of the algebra consists of the matrices $M_i$ with elements:
\[(M_i)_j^{~k} = -iC_{ij}^{~~k},\]
where $C_{ij}^{~~k},$ are the structure constants. If the dimension of the Lie algebra is $n$, then this is the number of generators, and it also means that the dimension of the matrices $M_i$ is $n\times n$. 
% TODO: Comment on when index position is important (not for semi-simple compact groups). Source?
To prove that this is actually a representation, we need to check that the mapping from the elements of the algebra $X_i$ to the matrices $M_i$ satisfies the homomorphism criteria of the Lie algebra representations. From the Jacobi identity of the algebra it follows that
\begin{equation}
[M_i, M_j] = iC^{~~k}_{ij}M_k,
\label{eq:adjoint_rep_commutator}
\end{equation}
meaning that the matrices of the adjoint representation fulfils the same commutator relationship as the fundamental (generators). As a direct consequence the $M_i$ form a representation.

Note that the dimension $n$ of the matrices in the adjoint representation for $\mathfrak so(m)$ and $\mathfrak su(m)$, which is equal to the number of independent parameters, is $n=m(m-1)/2$ and $n=m^2-1$, respectively. For the matrices in the representation derived from the defining representation of the group the dimensions is the same as the dimension of the defining matrices, $m$. This means that in most cases the matrices in the adjoint representation is larger, with the exception of  $\mathfrak{so}(2)$ and  $\mathfrak{so}(3)$.

We can now briefly return to the $SU(2)$ and $SO(3)$ examples. These groups have the same structure constants because they have the same commutators between generators. Thus the adjoint representations of their algebras are the same. With structure constants $C^{~~k}_{ij}=\epsilon_{ij}^{~~k}$ we get the elements of the adjoint matrices $(M_i)_{jk} =-i\epsilon_{ijk}$, which gives $M_i=-iL_i$, with the $L_i$ being the matrices in (\ref{eq:SO3_generators}). 

From the adjoint representation of the Lie algebra we can construct  the adjoint representations of the Lie group. For example the adjoint representation of  $\mathfrak{su}(2)$ or $\mathfrak{so}(3)$ has the exponential map $\exp{(ia_i M^i)}=\exp{(a_i L^i)}$. This happens to generate the $SO(3)$ fundamental representations, but this is not generally the case, it is ``accidental'' that the dimensions of the fundamental representation of $SO(3)$, $n=3$, and the dimension of its adjoint representation $n=3(3-1)/2=3$ is the same.


%%%
\subsection{Dual representations}
\label{sec:dual_reps}

Every representation of a Lie algebra on a vector space $V$ (and every representation of a group) has a {\bf dual representation} (also known as a {\bf contragredient representation}) that is a representation on the {\bf dual vector space} $V^*$. The definition of a dual vector space is somewhat abstract, it is formally the space consisting of all linear maps $\phi:V\to K$, where $K$ is the field of $V$. Since the map takes members of $V$ and maps them linearly to the field $K$, for a finite-dimensional vector space we must be able to write any member of $V^*$ in terms of a row-vector over the same field, so that if $\mathbf v\in V$ and $\mathbf w^T \in V^*$, the map is given by the matrix multiplication $\mathbf w^T \mathbf v$ which gives a scalar. This means that in practice  $V^*$ consists of the corresponding row-vectors to the column vectors in $V$.\footnote{It is slightly unfortunate and confusing that the symbol for the dual space uses a star since this seems to imply complex conjugation, where instead we are actually talking about transpose, however, as we will see below, the complex conjugation is central to dual representations of unitary groups.}

The definition of the dual representation for a Lie algebra is as follows:\footnote{This definition also applies beyond finite-dimensional vector space with a basis since the transpose operation can be generalised beyond matrices.}
\df{If ${\mathfrak l}$ is a Lie algebra with a representation $\rho$ on a vector space $V$, then the {\bf dual representation} on the dual space $V^*$ is $\rho^*$ given by
\[ \rho^*(x) = -\rho(x)^T, \]
for all $x\in\mathfrak l$.}
To see why this makes sense we need to look at the exponential map of the dual representation. If $X_i$ is a generator in the representation $\rho$, then the dual representation $\rho^*$ tells us to use instead $-X_i^T$. In the exponentiation the group element $g=\exp(iX_i)$ becomes the group element $\exp(-iX_i^T)=\exp(-iX_i)^T=(g^{-1})^T$ with the dual representation. This is the only sensible representation that fulfils the group multiplication properties as matrices acting on row-vectors.
% For this to be a group representation we must have that a product of two group elements $(g_1^{-1})^T(g_2^{-1})^T$ 

For unitary groups like $U(n)$ or $SU(n)$ we have matrix representations of the Lie algebra with Hermitian generators $M_i=M_i^\dagger$,  generating the members in the group as $U=\exp(ia_iM^i)$. From the definition of the dual representation the group members generated in the dual representation are then
\[ U=\exp(-ia_iM^{iT})=\exp(-ia_i(M^{i\dagger})^*)=\exp(-ia_iM^{i*}), \]
meaning that the Lie algebra in the dual representation can be written in terms of Hermitian matrices $M_i'=-M_i^*$. As a check we can see that $M_i'$ indeed fulfils the same algebra:
\[ [M_i',M_j']=[M_i^*,M_j^*]=-iC_{ij}^{~~k}M_k^*=iC_{ij}^{~~k}M_k',\] 
which is true only if the structure constants $C_{ij}^{~~k}$ are real.\footnote{You can take this as a proof that the structure constants are real if you prove that the dual representation is actually a representation, or you can go the other way and prove that this is indeed a representation, if you can first show that the structure constants of unitary groups are real.}
For the matrix groups we denote the dual representations using a bar, so that for example the dual of a representation $\mathbf 2$ is $\mathbf {\bar 2}$. The dual representations play a special role in quantum field theories, as the known fermions and anti-fermions transform under a representation of the gauge group and its dual, respectively.

%If a finite-dimensional representation of a group is irreducible, then one can show that the dual representation is also irreducible. 
For some groups, such as $SU(2)$, the original and dual representations can be shown to be isomorphic. However, this is not in general the case. For $SU(3)$ the dual representations $\mathbf {\bar 3}$, $\mathbf {\bar 6}$, and $\mathbf{\overline{10}}$, are not isomorphic to $\mathbf 3$, $\mathbf 6$, and $\mathbf{10}$. On the other hand, $\mathbf 8$ and $\mathbf {\bar 8}$ are isomorphic.\footnote{However, the dual of the dual of any representation is always isomorphic to the original representation.} This becomes important in the Standard Model as the quarks transform under the $\mathbf 3$ representation of the $SU(3)_c$ gauge group, and the anti-quarks under $\mathbf {\bar 3}$.



%%%%%%%%%
\section{Exercises}
%%%%%%%%%

%%%
\begin{Exercise}[difficulty=1]
Show that $\mathbb{R}^3$ with the binary operator $[\mathbf x, \mathbf y]=\mathbf x \times \mathbf y$ is a Lie algebra.
\end{Exercise}
\begin{Answer}
Clearly,  $\mathbb{R}^3$ with the cross product is algebra since $\mathbb{R}^3$ is a vector space and the cross product maps two elements of $\mathbb{R}^3$ into $\mathbb{R}^3$. We thus need to check the Lie conditions: i) the cross product is bilinear $(a\mathbf x +b\mathbf y)\times \mathbf z = a\mathbf x\times\mathbf z + b\mathbf y\times \mathbf z$, ii) the cross products is anti-symmetric $\mathbf x \times \mathbf y=-\mathbf y \times \mathbf x$, and iii), 
\[
\mathbf x \times (\mathbf y \times \mathbf z ) + \mathbf y \times (\mathbf z \times \mathbf x )+ \mathbf z \times (\mathbf x \times \mathbf y ) = 0,
\] 
is a standard relationship for cross products.
\end{Answer}
%%%

\begin{Exercise}[]
Find the adjoint representation for $SU(2)$. Compare this to the fundamental representation of $SO(3)$.
\end{Exercise}

\begin{Exercise}[]
Show Eq.~(\ref{eq:adjoint_rep_commutator}).
\end{Exercise}



\begin{Exercise}[]
Let $A$ be an algebra based on a finite-dimensional vector space over a field $F$ ,with a basis $B=\{b_i | i=1,\ldots,n\}$. Show that the multiplication of elements in $A$ is completely determined by the $n^2$ products $b_ib_j$ for each pair of basis vectors in $B$. 
\end{Exercise}

\begin{Exercise}[]
Let $V$ be a finite-dimensional vector space over a field $F$ with a basis $B=\{b_i | i=1,\ldots,n\}$. Let  $\{c_{rst}| r,s,t=1,\ldots,n\}$ be a collection of $n^3$ elements in $F$. Show that there exists one, and only one, multiplication operation on $V$ so that $V$ is an algebra over $F$ under this multiplication and
\[ b_rb_s = c_{rst}b_t,\]
for every pair of basis vectors in $B$. The elements $c_{rst}$ are the structure constants of the algebra.
\end{Exercise}


\end{document}

